\documentclass[11pt]{report}

\usepackage{geometry}  
\geometry{letterpaper} 
\usepackage{graphicx}
%\usepackage[backend=bibtex]{biblatex}
\usepackage{array}
\usepackage{amssymb}
\usepackage{amsmath}
\usepackage{amsthm}
\usepackage{graphicx}
\usepackage[parfill]{parskip} 
\usepackage[utf8]{inputenc}
\usepackage[english]{babel}
\usepackage{tikz}
\usepackage{tikz-cd}
\usepackage[noend]{algpseudocode}
\usepackage{caption}
\usepackage{subcaption}
\usepackage{fancyhdr}
\usepackage{enumitem}
\usepackage[super]{nth}
\usepackage{pstricks}

\usepackage[colorlinks=true,linkcolor=blue]{hyperref}


\pagestyle{fancy}
\lhead{}
\chead{}
\rhead{}
\lfoot{}
\cfoot{\thepage}
\rfoot{}
\renewcommand{\headrulewidth}{0pt}
\setlength{\footskip}{50pt}

\makeatletter
\def\BState{\State\hskip-\ALG@thistlm}
\makeatother

\theoremstyle{definition}
\newtheorem{thm}{Theorem}[section]
\theoremstyle{definition}
\newtheorem{cor}[thm]{Corollary}
\theoremstyle{definition}
\newtheorem{prop}[thm]{Proposition}
\theoremstyle{definition}
\newtheorem{dfn}[thm]{Definition}
\theoremstyle{definition}
\newtheorem{lem}[thm]{Lemma}
\theoremstyle{definition}
\newtheorem{ex}[thm]{Example}
\theoremstyle{definition}
\newtheorem{conj}[thm]{Conjecture}
\theoremstyle{definition}
\newtheorem*{rem}{Remark}

\newcommand{\Rom}[1]
    {\MakeUppercase{\romannumeral #1}}
\newcommand{\C}[1]{(\mathbb{C}^*)^#1}
\newcommand{\ldp}{log del pezzo }
\newcommand{\mb}[1]{\mathbb{#1}}
\newcommand{\Hi}{Hirzebruch surface }
\newcommand{\minres}{minimal resolution }
\newcommand{\LJ}{Looijenga pair }
\newcommand{\ra}{\rightarrow}

\graphicspath{ {images/} }

\begin{document} 

This is a short document to illustrate where I am with my thesis.
\\
\\
Setup of thesis by chapters:
\\
Chapter 0: Background/  Introduction \\
Chapter 1: Toric degenerations of log extremal extractions. \\
Chapter 2: A Log del pezzo with $h^0(-K_X) = 4$ and no toric degenerations. \\
Chapter 3: Log del Pezzo's with a $\mathbb{C}^*$ action. \\
Chapter 4: Terminal Fano 3-folds with a ${\mathbb{C}^*}^2$ action. \\
\\
\\
Progress by chapter and relevance in the field: 
\\
 Chapter one is a fairly straight forwards series of calculations. This is currently written up very badly and needs to be edited but it is written up. This is strongly related to one other paper and can be used to recreate the results of this paper in a different way in dimension 2.
\\
Chapter two is mainly focused on a single example which provides a counterexample to a conjecture in a paper of 2015. The example is written out, but the full proof is not, although the full proof is mainly just reading out computer outputs. This may be merged with chapter one for cohesion. We may put here some results which are relevant but with nowhere better to put them. These have been partially written up.
\\
Chapter three is a generalization of a two separate papers. It has been completely written up, however should be rewritten as the original version was very poorly written. This was submitted as my fourth year report. This chapter requires code. The code is simple and should take a very short amount of time to implement, and has been implemented in specific cases.
\\
Chapter four is an extension of a result of two papers. We provide a method that can be used to redo the first paper in a much more easy manner. We hope to extend this to the second paper as well. These papers aim to classify certain objects by a tree search. They classified tops of the tree, we show how you can obtain every element of the tree. This also involves code. The code is more involved than in chapter three. There is very little of this written up.
 

\end{document}