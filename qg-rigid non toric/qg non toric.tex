\documentclass[11pt]{amsart}

\usepackage{geometry}  
\geometry{letterpaper} 
\usepackage{graphicx}
%\usepackage[backend=bibtex]{biblatex}
\usepackage{array}
\usepackage{amssymb}
\usepackage{amsmath}
\usepackage{amsthm}
\usepackage{graphicx}
\usepackage[parfill]{parskip} 
\usepackage[utf8]{inputenc}
\usepackage[english]{babel}
\usepackage{tikz}
\usepackage{tikz-cd}
\usepackage[noend]{algpseudocode}
\usepackage{caption}
\usepackage{subcaption}
\usepackage{fancyhdr}
\usepackage{enumitem}
\usepackage[super]{nth}
\usepackage{pstricks}

\usepackage[colorlinks=true,linkcolor=blue]{hyperref}


\pagestyle{fancy}
\lhead{}
\chead{}
\rhead{}
\lfoot{}
\cfoot{\thepage}
\rfoot{}
\renewcommand{\headrulewidth}{0pt}
\setlength{\footskip}{50pt}

\makeatletter
\def\BState{\State\hskip-\ALG@thistlm}
\makeatother

\theoremstyle{definition}
\newtheorem{thm}{Theorem}[section]
\theoremstyle{definition}
\newtheorem{cor}[thm]{Corollary}
\theoremstyle{definition}
\newtheorem{prop}[thm]{Proposition}
\theoremstyle{definition}
\newtheorem{dfn}[thm]{Definition}
\theoremstyle{definition}
\newtheorem{lem}[thm]{Lemma}
\theoremstyle{definition}
\newtheorem{ex}[thm]{Example}
\theoremstyle{definition}
\newtheorem{conj}[thm]{Conjecture}
\theoremstyle{definition}
\newtheorem*{rem}{Remark}

\newcommand{\Rom}[1]
    {\MakeUppercase{\romannumeral #1}}
\newcommand{\C}[1]{(\mathbb{C}^*)^#1}
\newcommand{\ldp}{log del pezzo }
\newcommand{\mb}[1]{\mathbb{#1}}
\newcommand{\Hi}{Hirzebruch surface }
\newcommand{\minres}{minimal resolution }
\newcommand{\LJ}{Looijenga pair }
\newcommand{\ra}{\rightarrow}
\newcommand{\spl}{\text{SL}_2 (\mathbb{C})}
\newcommand{\gl}{\text{GL}_2 (\mathbb{C})}
\newcommand{\pgl}{\text{PGL}_2 (\mathbb{C})}

\graphicspath{ {images/} }

\begin{document} 

\section{The $\mathbb{Q}$g rigid singularities}

We wish to study groups $G$ that are finite subgroups $\gl$ with the property $G \cap \spl = \text{BD}_{4n}$. Consider the three elements
\[
\begin{array}{ccc}
a = \left(
\begin{array}{cc}
\zeta_{m} & 0  \\
0 & \zeta_{m} \\
\end{array} \right), &
b = \left(
\begin{array}{cc}
0 & 1  \\
-1 & 0 \\
\end{array} \right), &c_m =
\left( \begin{array}{cc}
\sigma_{2n} & 0 \\
0 & \sigma^{-1}_{2n}
\end{array} \right) 
\end{array}
\]

Here $\zeta$ We split this into two cases the case where $m$ is odd or even. In the odd case $G = \langle a, \, b, \, c_m \rangle$, otherwise it it is $\langle c, \, b \circ a_{2m} \rangle$.
Similar analysis can be done for the groups $E_7$ and $E_8$. We now discuss when their $\mathbb{Q}$g smoothings.
\\
\textbf{Case 1}
\\
Here $m$ is odd so  $G = \langle a, \, b , \, c \rangle$.  Looking at the index one cover we have a $x^2 + y^2z + z^{n+1}$. With $x = (u^{2n} - v^{2n})uv$, $y = u^{2n} + v^{2n}$, $z = u^2v^2$. Looking at the derivatives we see that $\mathcal{T}^1$ is generated by $1,y, z, \cdots z^{n-1}$. As this is a rational surface singularity we see that $\mathcal{T}^2 = 0$, so these deformations are unobstructed. We see that $c$ acts on $x,y,z$ with weights $(2n+2, 2n, 4)$. This correspond to $\frac{1}{m}(n+1, n, 2)$ action. Our equation has weight $2n+2$. To check rigidity we first need $n \not\cong_m -2$. In this case we have $y$ cannot be in the qG smoothing. Other than that we just need $z^i$ cannot have weight $2n+2$. This corresponds to $2n+2 \not\cong_m 2i$ for $i$ in $0 \dots n-1$. These will be the only qG rigid singularities. In particular this means that if $m > 2n+2$ then it is rigid.
\\
\\
\textbf{Case 2}
\\
Here $m = 2m'$ is even so $\langle a, \, b \circ c_{2m} \rangle$. So we have the same equation as above and we now get a 
$\frac{1}{2m'}(2n+m', 2n, 4)$, so to be rigid once again we need the $y$ term not to be in the smoothing $2n+4
 \not\cong_{2m'} 0$. Once again we need no $z^i$ terms this means  $4n+4 \not\cong_{2m'} 4i$  for $i$ in $0 \dots n-1$. 
 \\
 \\
 It is easy to see that if a partial smoothing exists then it either stay non cyclic quotient or, if you can deform equivariantly by $y$, it becomes a cyclic quotient singularity which is of the form $\frac{1}{m}(n+1, 2)$ or $\frac{1}{2m'}(2n_m, 4)$. From this it is easy to classify the $\mathbb{Q}g$ smoothable singularities. We note for a given $n$ there exists only a finite amount of $\mathbb{Q}g$ smoothable singularities with the $D_n$ singularity as its index one cover. This is in contrast to the $A_n$ case.
 
 \section{The $E_i$ Singularity}
 
 We start with the $E_6$ singularity as this behaves differently more similarly to the $D_n$ singularities. Once again we aim to classify the $\mathbb{Q}$G rigid singularities. We split it into two cases again. 
 \\
 \\
 \textbf{Case 1}
 \\
 This is a $\mathbb{Z}_n$ action with $n$ coprime to 6. This will act with weights $(6,4,3)$ on the equation $x^2 + y^3 + z^4$. This lies in the eigenspace of degree 12. Once again by computing partial derivatives we get that the smoothing parameters are $1, y, z, z^2$. We wish for none of these two be invariant. This means that $12 \not\cong_n 0, \, 4, \, 3, \, 6$. This implies that $n \neq  2, \, 3, \, 4, \, 9, \, 12$. However none of these cases can occur via the coprimality condition stated at the beginning.
 \\
 \\
 \textbf{Case 2}
 \\
 This is a $\mathbb{Z}_n$ action with $(n, 6) = 3$. Writing $n = 3n'$ we have the action with weights $(6 + n', 4, 3 + n')$. Once again this is non rigid if $n \neq  2, \, 3, \, 4, \, 9, \, 12$. Via the coprimality condition we get this is non rigid if $n=3$ or $n=9$.
 \\
 \\
 The $E_7$ and $E_8$ cases are not divide by case. The $E_7$ has a single possibility of $\mathbb{Z}_n$ with $(n,6) = 1$. This acts on $x^2 + y^3 + yz^3$ with the weights $(9,6,4)$. The partials give us a basis for the deformations of $1, y, y^2, z, z^2, yz$. This means that $18 \not\cong_n 0, \, 6, \, 12, \, 4, \, 8, \, 10$. All of these would correspond to $n$ being even so there are no non rigid singularities. 
 
 
 \section{Singularity content}
 
 We wish to generalise the notion of singularity content from cyclic quotients. We have that given a non cyclic quotient there exists a unique residual singularity. We wish to have other invariants indicating what it has smoothed from. The natural thing is to consider the invariant $h^2_{\text{top}}(X^0)$ where $X^0$ is $X-\{\text{singular locus}\}$. First considering the case where there is no smoothing in terms of $y$ so the only terms we can put in are of the form $z^i$ giving $x^2 + y^2z +z^j \Pi_{k=1}^{p} (z^i - a_k)$ then it is clear that we have exactly by projecting $(x,y,z) \mapsto z$ we get a map to $\mathbb{C}$ with degenerate fibers of 2 curves transversely intersecting over the finitely many points corresponding to $\lfloor \frac{n+1}{p} \rfloor$. 
 \textbf{Ignoring the central fiber over the origin}. We can apply Mayer-Vietoris repeatedly to get that $h^2_{\text{top}}(X^0)$ is a deformation invariant. In the case where we have $y$ in the deformation we have 
 

\end{document}