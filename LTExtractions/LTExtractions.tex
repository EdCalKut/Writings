\documentclass[11pt]{report}

\usepackage{geometry}  
\geometry{letterpaper} 
\usepackage{graphicx}
%\usepackage[backend=bibtex]{biblatex}
\usepackage{array}
\usepackage{amssymb}
\usepackage{amsmath}
\usepackage{amsthm}
\usepackage{graphicx}
\usepackage[parfill]{parskip} 
\usepackage[utf8]{inputenc}
\usepackage[english]{babel}
\usepackage{tikz}
\usepackage{tikz-cd}
\usepackage[noend]{algpseudocode}
\usepackage{caption}
\usepackage{subcaption}
\usepackage{fancyhdr}
\usepackage{enumitem}
\usepackage[super]{nth}
\usepackage{pstricks}

\usepackage[colorlinks=true,linkcolor=blue]{hyperref}


\pagestyle{fancy}
\lhead{}
\chead{}
\rhead{}
\lfoot{}
\cfoot{\thepage}
\rfoot{}
\renewcommand{\headrulewidth}{0pt}
\setlength{\footskip}{50pt}

\makeatletter
\def\BState{\State\hskip-\ALG@thistlm}
\makeatother

\theoremstyle{definition}
\newtheorem{thm}{Theorem}[section]
\theoremstyle{definition}
\newtheorem{cor}[thm]{Corollary}
\theoremstyle{definition}
\newtheorem{prop}[thm]{Proposition}
\theoremstyle{definition}
\newtheorem{dfn}[thm]{Definition}
\theoremstyle{definition}
\newtheorem{lem}[thm]{Lemma}
\theoremstyle{definition}
\newtheorem{ex}[thm]{Example}
\theoremstyle{definition}
\newtheorem{conj}[thm]{Conjecture}
\theoremstyle{definition}
\newtheorem*{rem}{Remark}

\newcommand{\Rom}[1]
    {\MakeUppercase{\romannumeral #1}}
\newcommand{\C}[1]{(\mathbb{C}^*)^#1}
\newcommand{\ldp}{log del pezzo }
\newcommand{\mb}[1]{\mathbb{#1}}
\newcommand{\Hi}{Hirzebruch surface }
\newcommand{\minres}{minimal resolution }
\newcommand{\LJ}{Looijenga pair }
\newcommand{\ra}{\rightarrow}

\graphicspath{ {images/} }

\begin{document} 


\section{Log extremal cyclic extractions}

We wish to classify the log extremal cyclis of a given log terminal singularity. We start with the case of a cyclic quotient singularity. Given a cyclic quotient singularity $S$ with resolution $a_1, \dots a_n$. Here we allow we separate the case $n=0$ which corresponds to $\mathbb{A}^2$. Let $\widetilde{S}$ be the minimal resolution of $S$.

\begin{lem}
Given a cyclic quotient singularity $S \neq \mathbb{A}^2$ with resolution $a_1, \dots a_n$. Then any cyclic extraction has curve configuration 

\begin{tikzpicture}
\matrix[matrix of math nodes,row sep ={1cm,between origins},column sep={1.5cm,between origins}] (m) {
a_1' & a_2' & \dots & a_i' + k & \dots & a_{n'}' \\
& & & 1 & & \\
& & & 2 & & \\
& & & \vdots & & \\
& & & 2 & & \\
};
  \path[thick]
(m-1-4) edge (m-2-4)
(m-1-1) edge (m-1-2)
(m-2-4) edge (m-3-4);
    
\end{tikzpicture}
\end{lem}

\begin{proof}
Let $\widetilde{S}$ be the minimal resolution of $S$. Let $\pi : X \rightarrow S$ be the log extraction, with $\widetilde{X}$ being the minimal resolution. Then $\widetilde{X}$ is constructed as a sequence of blowups of $\widetilde{S}$. As $\rho(X) = 1 = \rho(S) + 1$. This means that $\widetilde{X}$ has one $-1$ curve. Hence every blowup, except the first, is at a point on the $-1$ curve. Our sequence of blowups can be split into the ones that keep all the torus actions and the ones that don't. After the toric blowups we are are left with a configuration $a_1' \dots a_{n'}'$. Assuming the first blowup non toric is centered on a point  lieing on one of the negative curves, call this the $a_i$ curve. If our next blowup is not the intersection of the $-1$ curve and one of the$a_i$ curve then the $a_i$ curve is trivalent, and our resulting singularity will clearly not be cyclic quotient. Hence there is only one choice of point we can blowup leading to the above curve configuration. Note that there are no non toric blowups the choice of curve is  upon which the non toric blowups are centred is fixed .
\end{proof}

\begin{rem}
In the case of $\mathbb{A}^2$ a similar process occurs. Once again there is a toric part, which leaves us with 
$a_1, \dots a_n, 1, b_1, \dots b_m$. The same process as above happens obtaining the curve configuration

\begin{tikzpicture}
\matrix[matrix of math nodes,row sep ={1cm,between origins},column sep={1.5cm,between origins}] (m) {
a_1' & a_2' & \dots & a_n & 1 + k & b_1 & \dots & b_m \\
& & & & 1 & & & \\
& & & & 2 & & & \\
& & & & \vdots & & & \\
& & & & 2 & & & \\
};
  \path[thick]
(m-1-5) edge (m-2-5)
(m-1-1) edge (m-1-2)
(m-2-5) edge (m-3-5)
(m-1-4) edge (m-1-5)
(m-1-5) edge (m-1-6);
    
\end{tikzpicture}

\end{rem}

Now given a non cyclic quotient singularity a similar process occurs. However rather than a Du Val singularity trailing out you have a singularity of higher index. This implies that these do not have equivariant toric degenerations. We can write all of these extractions down as complexity one varieties. In the cyclic case they are formed by the glueing of two polyhedral divisors 
\[
\begin{array}{l}
\mathcal{D}_1 = [\frac{a}{b}, \infty) \otimes P_1 + [\frac{c}{d}, \infty) \otimes P_3 \\
\mathcal{D}_2 =  \emptyset \otimes P_1 + [0, n] \otimes P_2 + \emptyset \otimes P_3
\end{array}
\]
and in the non cyclic case 
\[
\begin{array}{l}
\mathcal{D}_1 = [\frac{a}{b}, \infty) \otimes P_1 + [n, \infty) \otimes P_2 + [\frac{c}{d}, \infty) \otimes P_3 \\
\mathcal{D}_2 =  \emptyset \otimes P_1 + [\frac{e}{f}, n] \otimes P_2 + \emptyset \otimes P_3
\end{array}
\]
All polyhedral divisors are over a base curve of $\mathbb{P}^1$. We see that if it admitted an equivariant toric degeneration there would be a way of combining two of the $P_i$. However the condition for it to be a valid deformation is that at least one of the two ends of the intervl are integers, and this cannot happen in the non cyclic quotient case. In the case arising from a cyclic quotient singularity we can take the minkowski sum of the polyhedral divisors $\frac{a}{b} \otimes P_1$ with $[0, n] \otimes P_2$. This leads us to the toric configuration 
\[
\begin{array}{l}
\mathcal{D}_1 = [ n + \frac{a}{b}, \infty) \otimes P_1 + [\frac{c}{d}, \infty) \otimes P_3 \\
\mathcal{D}_2 =  [\frac{a}{b}, n + \frac{a}{b}] \otimes P_1
\end{array}
\]
\newline
\begin{thm}
When it exists, given the above toric degeneration of a log terminal extraction then the map extends to a map of 3-folds. With $\mathcal{X} \rightarrow \mathbb{P}^1 \times S$.
\end{thm}
\vspace{0.1cm}
Given the assumptions above we can assume that $S$ is a $\frac{1}{r}(1,s)$ singularity and has a fan $\Sigma$ with rays $(a,b)$ and $(c-kd,-d)$ with $a,b,c,d  > 0$. We can also assume that the ray $(1,0)$ corresponds to $E$ in the minimal resolution. We also note that $r$ is equal to the determinant of the rays, $ad - b(c-kd)$. 
\\
\\
We note that $X$ has a torus acting on it. This means that it can be written as a polyhedral divisor $\mathcal{S}$.
\begin{figure}[htbp]
\psset{unit=0.95cm}
\begin{pspicture}(0,-6)(12,0)
%\psgrid(0,0)(0,-8)(12,0)
\psframe[linecolor=white](0.5,-4.5)(3.5,-1.5)


\psline{<-|}(6,-3)(7.5,-3)
\psline{|->}(7.5,-3)(10,-3)
%\uput*[270](8,-3){$0$}
\rput[bl]{0}(11,-3){$\mathcal{S}_1$}
\uput*[270](7.6,-3.1){${\tiny \frac{a}{b}}$}

\psline{<-|}(6,-1.8)(7.25,-1.8)
\psline{-|}(7.25,-1.8)(8,-1.8)
\psline{->}(8,-1.8)(10,-1.8)
\uput*[270](7.2,-1.9){${\tiny {0}}$}
\uput*[270](8,-1.9){${\tiny{k}}$}
\rput[bl]{0}(11,-1.8){$\mathcal{S}_0$}
\rput[bl]{0}(9,-1.6){$\mathcal{D}_{\sigma_0}$}
\rput[bl]{0}(7.3,-1.6){$\mathcal{D}_{\sigma_1}$}


\psline{<-|}(6,-4.2)(9,-4.2)
\psline{|->}(9,-4.2)(10,-4.2)
\uput*[270](8.9,-4.3){${\tiny \frac{c-kd}{d}}$}
\rput[bl]{0}(11,-4.2){$\mathcal{S}_{\infty}$}
\rput[bl]{0}(9,-4){$\mathcal{D}_{\sigma_0}$}
\rput[bl]{0}(8,-5.5){$\mathcal{S}$}

\psline{|->}(5,-3)(5,-1)
\psline{|->}(5,-3)(5,-5)
%\qdisk(5,-1.8){1pt}
%\qdisk(5,-4.2){1pt}
%\qdisk(5,-3){1.5pt}
\rput[bl]{0}(4.5,-5.5){$Y=\mathbb{P}^1$}


\end{pspicture}
\caption{Divisorial fan associated to $X$.}
\end{figure}

Here the tail fan is $\mathbb{Z}_{\geq 0}$. When constructing the equivariant toric degeneration, we take the Minkowski sum of $\frac{c-kd}{d}$ and the cone $\left[0, k\right]$. After summation this is the toric variety with rays $(a,b)$, $(c,-d)$, $(c-kd,-d)$. Call this $X_\Sigma$. The cone $(c,-d)$, $(c-kd,-d)$ is a $T$-singularity. The cone  $(a,b)$, $(c,-d)$ is a $\frac{1}{t}(1, u)$ singularity with once again $t = bc + ad$. Labeling these three rays $v_1, v_2, v_3$ we get the relation $kd^2 v_1 - r v_2 + t v_3 = 0$. This implies that the Cox Ring  $\mathcal{R}(X_\Sigma)$ is $\mathbb{A}^3_{<x,y,z>}$ with a quotient of $\mathbb{C}^*$ with weights $(kd^2, -r, t)$ and a finite group action $\mathbb{\mu}$. It is easy to show that $|\mathbb{\mu}| = \text{hcf(b,d)} = e$ as $(x, ey)$ is clearly a sub lattice which contains all the vertices and $(1,0)$ is in the lattice as $b$ and $d$ are coprime. This is a well defined quotient except along $x = z= 0$. To construct the deformation, we take the $d$-fold veronese embedding getting
\[
\frac{\mathbb{C}[y_1, y_2, y_3,  y_4]}{y_2^d -y_3 y_4} \text{  with weights } (kd, kb, -r, t)
\]

Here the $b$ occurs as $t-r = kdb$. We can now construct the deformation family by the following equation

\[
\frac{\mathbb{C}[y_1, y_2, y_3,  y_4]}{\lambda y_1^b  + \mu y_2^d - y_3 y_4} \text{  with weights } (kd, kb, -r, t)
\]

At $\lambda = 0$ this is our original variety. At $\mu = 0$ we get the variety corresponding to the Minkowski sum of the cones $\frac{a}{b}$ and the cone $\left[ 0, 1 \right]$. This is the toric variety with rays $(c-d,-d), \, (a+b, b), \, (a,b)$. Note that the index of the sublattice generated by these rays is still equal to hcf$(b, d)$. All that remains is to show that this is the desired complexity one variety. From the above polyhedral divisor we get that $\mathcal{R}(X) =\frac{\mathbb{C}[y_1, y_2, y_3,  y_4]}{\lambda y_1^b  + \mu y_2^d - y_3 y_4}$. We now have to calculate the weights of the torus action on it. The Weil divisors on $X$ satisfy the relations given by the rows of the following matrix. 
\[
 \left(
 \begin{array}{cccc}
b & 0 & -1 & -1  \\
0 & c & -1 & -1 \\
a & d & 0  & k \\
\end{array}
\right) 
\]
To calculate the grading is equivalent to finding a relation between the columns of the matrix. It is easy to verify that the weights $kd,kb,-r,t$ give the corresponding column sum of 0. To verify that the same finite group acts on the fiber we see that if we do Gaussian elimination. We get a row $(d, -b, 0 ,0)$ if these two numbers are not coprime this means that we have torsion in the class of size hcf$(b,d)$. So $\mathbb{\mu}$ action extends across the family. 
\\
\\
We now classify cyclic log fano extractions on surfaces.
\begin{lem}
Let $f: \, X \rightarrow S$ be a cyclic log fano extarction. Then $S$ has to be a cyclic singularity and $f$ has to be the map as described above. This results in potentially two cyclic quotient singularities. One of them is a general $\frac{1}{r} (1, a)$ and the other is of type $A_n$.
\end{lem}
\end{document}