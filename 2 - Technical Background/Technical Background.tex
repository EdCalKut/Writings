\documentclass[12pt]{amsbook}

\usepackage{geometry}  
\geometry{letterpaper} 
\usepackage{graphicx}
%\usepackage[backend=bibtex]{biblatex}
\usepackage{array}
\usepackage{amssymb}
\usepackage{amsmath}
\usepackage{amsthm}
\usepackage{graphicx}
\usepackage[parfill]{parskip} 
\usepackage[utf8]{inputenc}
\usepackage[english]{babel}
\usepackage{enumerate}
\usepackage{tikz}
\usepackage{tikz-cd}
\usepackage[noend]{algpseudocode}
\usepackage{caption}
\usepackage{subcaption}
\usepackage{fancyhdr}
\usepackage{enumitem}
\usepackage[super]{nth}
\usepackage{pstricks}
\usepackage{xstring}
\usepackage{comment}
\usepackage{pgfplots}
\usepackage{blkarray} %This is for labelled matrices
\usetikzlibrary{decorations.pathreplacing} % this is for Ai's drawing



\usepackage[colorlinks=true,linkcolor=blue]{hyperref}


\pagestyle{fancy}
\lhead{}
\chead{}
\rhead{}
\lfoot{}
\cfoot{\thepage}
\rfoot{}
\renewcommand{\headrulewidth}{0pt}
\setlength{\footskip}{50pt}

\makeatletter
\def\BState{\State\hskip-\ALG@thistlm}
\makeatother

\theoremstyle{plain}
\newtheorem{thm}{Theorem}[section]

\newtheorem{cor}[thm]{Corollary}

\newtheorem{prop}[thm]{Proposition}

\newtheorem{dfn}[thm]{Definition}

\newtheorem{lem}[thm]{Lemma}

\newtheorem{ex}[thm]{Example}

\newtheorem{conj}[thm]{Conjecture}

\newtheorem*{rem}{Remark}
\newtheorem{assumption}[thm]{Assumption}

\newcommand{\Rom}[1]
    {\MakeUppercase{\romannumeral #1}}
\newcommand{\C}[1]{(\mathbb{C}^*)^#1}
\newcommand{\ldp}{log del Pezzo}
\newcommand{\mb}[1]{\mathbb{#1}}
\newcommand{\Hi}{Hirzebruch surface }
\newcommand{\minres}{minimal resolution}
\newcommand{\LJ}{Looijenga pair}
\newcommand{\ra}{\rightarrow}
\newcommand{\spl}{\text{SL}_2 (\mathbb{C})}
\newcommand{\gl}{\text{GL}_2 (\mathbb{C})}
\newcommand{\pgl}{\text{PGL}_2 (\mathbb{C})}
\newcommand{\wt}[1]{\widetilde #1}
\newcommand{\Q}{\mathbb{Q}}
\newcommand{\Z}{\mathbb{Z}}
\newcommand{\F}{\mathbb{F}}
\renewcommand{\P}{\mathbb{P}}

% Declare your maths operators

\graphicspath{ {images/} }

\begin{document} 

\setcounter{chapter}{1}
Nothing in this chapter is original work, and references are provided. Throughout this thesis we work only over the field $\mb{C}$.
\section{Toric Geometry}

We use the traditional language set up in \cite{cox}. In particular we use the fact that a normal toric variety $X$ of dimension $n$ can be associated with a (non unique) fan $\Sigma \subset N \cong \mb{Z}^n$. We consider the dual lattice to $N$, denoted $M$. Any $m \in M$ corresponds to a character of the torus which in turn corresponds to a monomial function in the function field of $X$. We denote the 1-skeleton of one dimension cones in $\Sigma$ by $\Sigma^1$. We also use this to refer to the set of primitive vectors generating those rays; this is a small abuse of notation that is always clear in context. We say a fan is complete if every lattice point $u \in N$ lies inside some cone $\sigma \subset \Sigma$. The associated variety to a complete fan is complete.
\subsection{Cox rings}
Given a toric variety $X$, we wish to construct it as a GIT quotient. We follow the construction of \cite{cox}. Given a complete fan $\Sigma$ with $\Sigma^1 = \{ v_1, \, \dots , \, v_m\}$, we consider the toric variety given by a fan $\overline{\Sigma} \subset \mb{Z}^m$, with
$\overline{\Sigma}^1 = \{e_i\}$, where $e_i$ are the standard basis vectors. A set $\{e_i\}_{i \in S}$ spans a cone in $\overline{\Sigma}$ if and only if the set $\{v_i \}_{i \in S}$ spans a cone of $\Sigma$. The variety $Y$ associated to $\overline{\Sigma}$ is a subset of~$K^m$. By construction we have a well defined map of fans $\phi \colon \overline{\Sigma} \ra \Sigma$ corresponding to a linear projection. This induces a map $\tilde{\phi} \colon Y \ra X$ which can be seen as a GIT quotient with weights corresponding to the linear dependencies of $\Sigma^1$, and a finite group corresponding to the index of the sublattice of $N$ generated by $\Sigma^1$.

\subsection{Cyclic quotient singularities and singularity content}

We also make frequent use of the following concepts introduce in \cite{Reid-cyclic} and \cite{SingContent}. Suppose given a cyclic quotient singularity $S=\frac{1}{r}(a,b)$ in two dimensions. Here $S$ is the quotient of $\mb{C}^2$ by the group $G \cong \frac{\mb{Z}}{r\mb{Z}}$, with action defined by the matrix
\[
\left(
\begin{array}{cc}
\zeta^a & 0 \\
0 & \zeta^b \\
\end{array}
\right)
\]
where $\zeta = e^{\frac{2 \pi i}{r}}$. Without loss of generality $a$ and $b$ are coprime to $r$. This in turn implies that, by change of basis, we can write $S$ as $\frac{1}{r}(1,u)$. The minimal resolution of this singularity is a chain of curves $C_1$, $\dots$, $C_n$ with self intersections equal to $[a_1,\, \dots , a_n]$, where these values $a_i$ are equal to the coefficients of the Hirzebruch Jung continued fraction of $\frac{r}{u}$, as laid out in \cite{Reid-cyclic}.


We are mainly interested in studying the restricted class of deformations know as $\mb{Q}$-Gorenstein as given in \cite{Kollar-SB}. 
\begin{dfn}

\end{dfn}
Singularity content is a concept introduced in \ref{SingContent} as a $\mb{Q}$-Gorenstein deformation invariant of a surface. Given a surface singularity $S$ we define the index one cover $S_1$ to be the quotient of $\mb{C}^2$  by the subgroup $H = G \cap SL_2(\mb{Z})$. This gives $\mb{C}^2 \ra S_1 \ra S$ where $S_1$ has a singularity of type $A_n$, and this has equation $xy = z^{n+1}$. The group $G/H$ acts on $S_1$ with quotient $S$. That is, this group acts on $xy = z^{n+1}$ with some weight $k$; this means $G/H \cong \frac{\mb{Z}}{\frac{r}{n}\mb{Z}}$ acts naturally by  some weights on the $x$, $y$, $z$ and the equation has weight $k$. This gives us the $\mb{Q}$-Gorenstein deformations  of $S$ are  the quotients of the equivariant deformations $xy = \sum a_i z^{k + i\frac{r}{n}}$. This is smooth if and only if $k=0$. On the other hand if $k\neq 0$ the deformation has a residual singularity $\frac{1}{r'}(a', b')$. We call the pair $(n, \, \frac{1}{r'}(a', b')) $ the singularity content. If $n=0$ we say the singularity is $\mb{Q}$-Gorenstein rigid. The value $n$ can be seen to be equal to the topological Euler number of the $\mb{Q}$-Gorenstein smoothing with the singular point removed, although this is not used in this thesis.


Given a log del Pezzo surface $X$ with only $\mb{Q}$-Gorenstein rigid singularities, we define the singularity content $(n, \{S_1, \, \dots, S_n\})$ where $S_i$ are the singularities of $X$ and $n$ is once again the topological Euler number of $X^0 = X - \{\text{Singular locus of } X\}$. In \cite{Section 4} we show how the value $n$ fits into the language of affine manifolds.
\section{Log del Pezzo background}

\subsection{Definitions}
We here relate some basic definitions and facts about surfaces.

Given a normal surface singularity $S$ and minimal resolution $\pi \colon \wt{S} \ra S$ then we have 
\[
K_{\wt{S}} = \pi^*(K_S) + \sum a_i E_i
\] 
\begin{dfn}
Throughout this thesis a log del Pezzo surface is a normal complex projective surface with log terminal singularities and ${-}K_X$ ample.
\end{dfn}
Where we say a singularity is 
\begin{itemize}
\item terminal singularities if $a_i > 0$
\item canonical singularities if $a_i \geq 0$
\item log terminal singularities if $a_i \geq 0$
\item log canonical singularities is $a_i \geq 1$
\end{itemize}
A surface singuarity is log terminal if and only if it can be constructed as a quotient of $\mb{C}^2$  by a, not necessarily cyclic, group action \ref{Kawamata}. The classification of smooth log del Pezzo surfaces have been classified as the blowups of $\mb{P}^2$ at less than $9$ general points.

Given an orbifold log del Pezzo surface we frequently use the invariants 
${-}K_X^2$ and $h^0({-}K_X)$. These can be via orbifold Riemann Roch as set out in \ref{YoungPersonGuide}. For a rough sketch of how we do these calculations, given a singular $X$, with minimal resolution $Y$. Then ${-}K_Y^2$ and $h^0({-}K_Y)$. To account for these the contractions, there is a correction term which we calculate via toric geometry, in the case of case of ${-}K_X^2$ this corresponds to the area of lattice cones contained in $N$ corresponding to the singularities and in the case of $h^0({-}K_X)$ this corresponds to a count of lattice points in the dual of the cone inside the lattice $M$.  These are invariant under $\mb{Q}$-Gorenstein deformation. 
 
\subsection{Hirzebruch Surfaces} % Park city, alternatively toric geometry. 
We briefly state some basic results about Hirzebruch surfaces \ref{Park city}. A Hirzebruch surface is a rational scroll defined as the quotient of $\mb{C}^4$ by $(\mb{C}^*)^2$ with weights $(1,-1,0,0)$ and $(n, 0, 1,1)$. Alternatively it is the minimal resolution of $\mb{P}(11n)$. From this we see that we have the picard group generated by $B$ and $F$, where $B^2 = -n$ and $F$ is a fiber of the map to $\mb{P}^1$. From this it is straight forwards to see the possible smooth rational curves on a Hirzebruch surface, let $A = B+ nF$ then every smooth reduced rational curve lies in one of the linear equivalence classes $|A|$, $|2A|$, $|A+F|$, $|A+2F|$, $|B|$, $|F|$ and $|2F|$. 
\subsection{Basic Surfaces}
% Cascades are not needed. 
We finish with a very brief overview of \ref{CortiHeu}, \ref{Cuzzucoli} and \ref{CaveyPrince} as some of the methods we employ are similar. Respectively these paper classify log del Pezzos with singularities with minimal resolution $[3]$ in \ref{CortiHeu}, $[3,2]$ and $[3]$ in \ref{Cuzzucoli}, and finally one singularity with resolution $[n]$ in \ref{CaveyPrince}. The structure is similar, classify the possible surfaces $X$ which admit no Mori contractions to another surface which could arise from these choices of singularities. These are called basic surfaces. Then study their blowups and their birational relations, often in the context of cascades as introduced by \ref{ReidSuzuki}. Via these explicit classifications they have been able to give explicit coordinate contstructions and their toric degenerations (when they exist). In Chapter 2 we classify log del Pezzo surfaces with singularities of the form $[a_1, \, \dots ,   \, a_n]$ with both $a_1$ and $a_n$ greater than two via similar although modified methods.


\section{Gross Siebert}

In this section we do not use or even refer to the full power of the Gross-Siebert program, we are mainly using results referencing how certain SYZ fibrations are constructed and how these give rise to toric degenerations. Our main reference throughout is \ref{GrossBook}, mainly chapters 1.4 and 1.6. We have made a variety of small changes to notation, namely our fans lie inside $N$ instead of $M$ to be consistent with the rest of the notation within the thesis and we do not consider fans on non compact affine manifolds.

We start with the definition of a tropical affine manifold
\begin{dfn}

\end{dfn}
and then allow singularities on this tropical manifold with the following definition:
\begin{dfn}

\end{dfn}
We finish by extending the definition of a fan to lie on tropical affine manifolds with singularities. This allows us to have a one to one correspondence between these fans on tropical manifolds with singularities and surfaces $X$ such that the minimal resolution $Y$ admits a map to a toric variety which is a series of blowups along curves on the boundary. 
\end{document}

