\documentclass[11pt]{report}

\usepackage{geometry}  
\geometry{letterpaper} 
\usepackage{graphicx}
%\usepackage[backend=bibtex]{biblatex}
\usepackage{array}
\usepackage{amssymb}
\usepackage{amsmath}
\usepackage{amsthm}
\usepackage{graphicx}
\usepackage[parfill]{parskip} 
\usepackage[utf8]{inputenc}
\usepackage[english]{babel}
\usepackage{tikz}
\usepackage{tikz-cd}
\usepackage[noend]{algpseudocode}
\usepackage{caption}
\usepackage{subcaption}
\usepackage{fancyhdr}
\usepackage{enumitem}
\usepackage[super]{nth}
\usepackage{pstricks}

\usepackage[colorlinks=true,linkcolor=blue]{hyperref}


\pagestyle{fancy}
\lhead{}
\chead{}
\rhead{}
\lfoot{}
\cfoot{\thepage}
\rfoot{}
\renewcommand{\headrulewidth}{0pt}
\setlength{\footskip}{50pt}

\makeatletter
\def\BState{\State\hskip-\ALG@thistlm}
\makeatother

\theoremstyle{definition}
\newtheorem{thm}{Theorem}[section]
\theoremstyle{definition}
\newtheorem{cor}[thm]{Corollary}
\theoremstyle{definition}
\newtheorem{prop}[thm]{Proposition}
\theoremstyle{definition}
\newtheorem{dfn}[thm]{Definition}
\theoremstyle{definition}
\newtheorem{lem}[thm]{Lemma}
\theoremstyle{definition}
\newtheorem{ex}[thm]{Example}
\theoremstyle{definition}
\newtheorem{conj}[thm]{Conjecture}

\newcommand{\Rom}[1]
    {\MakeUppercase{\romannumeral #1}}
\newcommand{\C}[1]{(\mathbb{C}^*)^#1}
\newcommand{\ldp}{log del pezzo }
\newcommand{\mb}[1]{\mathbb{#1}}
\newcommand{\Hi}{Hirzebruch surface }
\newcommand{\minres}{minimal resolution }
\newcommand{\LJ}{Looijenga pair }
\newcommand{\ra}{\rightarrow}

\graphicspath{ {images/} }

\begin{document} 

We wish to show that given $(S, C) \rightarrow X$ a log terminal cyclic extraction on a normal affine toric surface then this admits a toric degeneration such that the contraction extends over the total space. This map can be constructed via the minimal resolution. If you take the minimal resolution $\widetilde{X}$ then there exists an exceptional curve $E$ such that the blowup of a general point on $E$ is the minimal resolution of $S$.
\\
\\
In particular we can assume that $X$ is a $\frac{1}{r}(1,s)$ singularity and has a fan $\Sigma$ with rays $(a,b)$ and $(c-d,-d)$ with $a,b,c,d  > 0$. We can also assume that the ray $(1,0)$ corresponds to $E$ in the minimal resolution. We also note that $r$ is equal to the determinant of the rays, $ad - b(c-d)$. 
\\
\\
We note that $S$ has a torus acting on it. Via the deformation theory of complexity one varieties we see that one of the equivariant toric degeneration is the toric variety with rays $(a,b)$, $(c,-d)$, $(c-d,-d)$. Call this $Y$. The cone $(c,-d)$, $(c-d,-d)$ is a $T$-singularity. The cone  $(a,b)$, $(c,-d)$ is a $\frac{1}{t}(1, u)$ singularity with once again $t = bc + ad$. Labelling these three rays $v_1, v_2, v_3$ we get the relation $d^2 v_1 - r v_2 + t v_3 = 0$. So writing out the Cox ring we have 
\[
\mathcal{R}(Y)  = \mathbb{C}[x_1, x_2, x_3] 
\]
With a $\mathbb{C}^*$ action with weights $(d^2, -r, t)$. Taking the $d$ fold veronese embedding of this gets us 

\[
\frac{\mathbb{C}[y_1, y_2, y_3,  y_4]}{y_2^d -y_3 y_4} \text{  with weights } (d, b, -r, t)
\]

Here the $b$ occurs as $t-r = db$. This lets us construct the deformation family by considering 

\[
\frac{\mathbb{C}[y_1, y_2, y_3,  y_4]}{\lambda y_1^b  + \mu y_2^d - y_3 y_4} \text{  with weights } (d, b, -r, t)
\]

All that remains is to show that this is the desired complexity one variety. This is equivalent to finding the Smith Normal Form of a a matrix 
\[
 \left(
 \begin{array}{cccc}
b & -1 & -1 & 0 \\
0 & -1 & -1 & d \\
a & 0 & 1 & c \\
\end{array}
\right) 
\]

\end{document}