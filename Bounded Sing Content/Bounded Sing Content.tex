\documentclass[11pt]{report}

\usepackage{geometry}  
\geometry{letterpaper} 
\usepackage{graphicx}
%\usepackage[backend=bibtex]{biblatex}
\usepackage{array}
\usepackage{amssymb}
\usepackage{amsmath}
\usepackage{amsthm}
\usepackage{graphicx}
\usepackage[parfill]{parskip} 
\usepackage[utf8]{inputenc}
\usepackage[english]{babel}
\usepackage{tikz}
\usepackage{tikz-cd}
\usepackage[noend]{algpseudocode}
\usepackage{caption}
\usepackage{subcaption}
\usepackage{fancyhdr}
\usepackage{enumitem}
\usepackage[super]{nth}
\usepackage{pstricks}

\usepackage[colorlinks=true,linkcolor=blue]{hyperref}


\pagestyle{fancy}
\lhead{}
\chead{}
\rhead{}
\lfoot{}
\cfoot{\thepage}
\rfoot{}
\renewcommand{\headrulewidth}{0pt}
\setlength{\footskip}{50pt}

\makeatletter
\def\BState{\State\hskip-\ALG@thistlm}
\makeatother

\theoremstyle{definition}
\newtheorem{thm}{Theorem}[section]
\theoremstyle{definition}
\newtheorem{cor}[thm]{Corollary}
\theoremstyle{definition}
\newtheorem{prop}[thm]{Proposition}
\theoremstyle{definition}
\newtheorem{dfn}[thm]{Definition}
\theoremstyle{definition}
\newtheorem{lem}[thm]{Lemma}
\theoremstyle{definition}
\newtheorem{ex}[thm]{Example}
\theoremstyle{definition}
\newtheorem{conj}[thm]{Conjecture}
\theoremstyle{definition}
\newtheorem*{rem}{Remark}

\newcommand{\Rom}[1]
    {\MakeUppercase{\romannumeral #1}}
\newcommand{\C}[1]{(\mathbb{C}^*)^#1}
\newcommand{\ldp}{log del pezzo }
\newcommand{\mb}[1]{\mathbb{#1}}
\newcommand{\Hi}{Hirzebruch surface }
\newcommand{\minres}{minimal resolution }
\newcommand{\LJ}{Looijenga pair }
\newcommand{\ra}{\rightarrow}
\newcommand{\spl}{\text{SL}_2 (\mathbb{C})}
\newcommand{\gl}{\text{GL}_2 (\mathbb{C})}
\newcommand{\pgl}{\text{PGL}_2 (\mathbb{C})}

\graphicspath{ {images/} }

\begin{document} 

\section{context}
\textbf{This is a first draft, context will be inserted}

\section{Result}

Throughout the rest of this chapter we limit ourselves to cyclic quotient singularities $S$ with the following property - let $C_1, \dots C_n$ be the minimal resolution of $S$, let the values $a_1 \dots a_n$ be the respective discrepancies, we insist that $a_i \leq \frac{1}{2}$. We call this a singularity with small discrepancy. This can be compared to other peoples work, in the following lemma
\begin{lem}
Let $S$ be a cyclic quotient singularity, $C_1 \dots C_n$ the minimal resolution. If $C_i^2 \leq -4$ then $S$ has small discrepancy.
\end{lem}
\begin{proof}
This is easy to see as $S$ can be described as a toric singularity with rays $(u_0, v_0), \, (u_{n+1},v_{n+1})$. Where $v_0 = v_{n+1} = h$ is the gorenstein index of $S$. A given curve  $C_i$ in the minimal resolution corresponds to a ray $(u_i,v_i)$ inside the above cone, so $v_i \leq h \forall i$. Now $(u_i, v_i) = \frac{(u_{i-1}, v_{n-1}) + (u_{i+1}, v_{i+1})}{C_i^2}$. So $v_i \leq \frac{\max(v_{i-1}, v_{i+1})}{2} \leq \frac{h}{2}$. The discrepancy of the curve $C_i$ is equal to $\frac{v_i-h}{h}$ which is clearly less than $\frac{-1}{2}$.
\end{proof}
We no explain why this makes classification so easy 
\begin{lem}
Let $X$ be a surface and  $f \: : Y \rightarrow X$ be the minimal resolution of $X$. Let $C \subset X$ be a curve that such that $C$ intersects two singularities (potentially the same) with small discrepancy. Consider the curve $\widetilde C \subset Y$ the strict transform. Then if $C^2 = -1$ then $-K_X \cdot C \leq = 0$.
\end{lem}
\begin{proof}
Let $f \: : Y \rightarrow X$ be the minimal resolution of $X$, $\widetilde C \subset Y$ the strict transform of $C$. As $C$ is a smooth curve on a smooth surface $-K_Y \cdot \widetilde C = 1$. We know that $C$ intersects at least two exceptional curve $E_i, \, E_j$, with discrepancy $a, \, b$. Via ??? we see that $-K_X \cdot C = f^*(-K_X) \cdot \widetilde C \leq 1 - a - b  \leq 0$. 
\end{proof}
Hence this curve configuration cannot lie on a \ldp. We also make the quick remark that in the case where the length $n$ of the singularity is 1 or 2, this remark follows via easy toric geometry as any curve joining two singularities is a locally toric configuration. This corresponds to the asscoiated fan being non convex. 
\begin{lem}
Let $X$ be a \ldp with only singularities of small discrepancy. As above let $f \: : Y \rightarrow X$ be the \minres then we consider the map $\pi \: : Y \rightarrow \mathbb{F}_l$. Consider $E$ exceptional curves in the minimal resolution. Then $\pi_* E$ is one of the following 
\begin{itemize}
\item A smooth curve with positive self intersection
\item A smooth curve with negative self intersection
\item A point
\end{itemize}
\end{lem}
\begin{proof}
This just amounts to it being impossible for $\pi_* E$ to be a non smooth curve. Hence assume that it has a singular point $P$. In order for us to get a cyclic quotient singularity, $E$ needs to be smooth. Hence there is a collection of curves $C_i \subset Y$ which blowdown to $P$. As all these curves are contracted $C_i^2 \leq = 0$ and $E^2 \leq 0$, and there is a curve $C_j$ with $C_j^2 = -1$. Clearly $C_j$ intersects either two curves with self intersection less than $-1$, or it could intersect $E$ twice. By the above lemma neither case could appear on the \minres of a \ldp.
\end{proof}
\end{document}