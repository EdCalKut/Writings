\documentclass[11pt]{report}

\usepackage{geometry}  
\geometry{letterpaper} 
\usepackage{graphicx}
%\usepackage[backend=bibtex]{biblatex}
\usepackage{array}
\usepackage{amssymb}
\usepackage{amsmath}
\usepackage{amsthm}
\usepackage{graphicx}
\usepackage[parfill]{parskip} 
\usepackage[utf8]{inputenc}
\usepackage[english]{babel}
\usepackage{enumerate}
\usepackage{tikz}
\usepackage{tikz-cd}
\usepackage[noend]{algpseudocode}
\usepackage{caption}
\usepackage{subcaption}
\usepackage{fancyhdr}
\usepackage{enumitem}
\usepackage[super]{nth}
\usepackage{pstricks}
\usepackage{xstring}

\usepackage[colorlinks=true,linkcolor=blue]{hyperref}


\pagestyle{fancy}
\lhead{}
\chead{}
\rhead{}
\lfoot{}
\cfoot{\thepage}
\rfoot{}
\renewcommand{\headrulewidth}{0pt}
\setlength{\footskip}{50pt}

\makeatletter
\def\BState{\State\hskip-\ALG@thistlm}
\makeatother

\theoremstyle{definition}
\newtheorem{thm}{Theorem}[section]
\theoremstyle{definition}
\newtheorem{cor}[thm]{Corollary}
\theoremstyle{definition}
\newtheorem{prop}[thm]{Proposition}
\theoremstyle{definition}
\newtheorem{dfn}[thm]{Definition}
\theoremstyle{definition}
\newtheorem{lem}[thm]{Lemma}
\theoremstyle{definition}
\newtheorem{ex}[thm]{Example}
\theoremstyle{definition}
\newtheorem{conj}[thm]{Conjecture}
\theoremstyle{definition}
\newtheorem*{rem}{Remark}

\newcommand{\Rom}[1]
    {\MakeUppercase{\romannumeral #1}}
\newcommand{\C}[1]{(\mathbb{C}^*)^#1}
\newcommand{\ldp}{log del pezzo }
\newcommand{\mb}[1]{\mathbb{#1}}
\newcommand{\Hi}{Hirzebruch surface }
\newcommand{\minres}{minimal resolution }
\newcommand{\LJ}{Looijenga pair }
\newcommand{\ra}{\rightarrow}
\newcommand{\spl}{\text{SL}_2 (\mathbb{C})}
\newcommand{\gl}{\text{GL}_2 (\mathbb{C})}
\newcommand{\pgl}{\text{PGL}_2 (\mathbb{C})}
\newcommand{\wt}[1]{\widetilde #1}


% This works both with inline lists and with macros containing lists
\newcommand*{\GetListMember}[2]{%
    \edef\dotheloop{%
    \noexpand\foreach \noexpand\a [count=\noexpand\i] in {#1} {%
        \noexpand\IfEq{\noexpand\i}{#2}{\noexpand\a\noexpand\breakforeach}{}%
    }}%
    \dotheloop
    \par%
}%


%4 variables #1 is xshift, #2 = #sides, #3 values, #4 num of curves, #5 is visible
\newcommand{\polygon}[5][xshift=0cm]{
  \pgfmathsetmacro{\angle}{360/#2}
  \pgfmathsetmacro{\startangle}{-90 + \angle/2}
  \pgfmathsetmacro{\y}{cos(\angle/2)}
  \begin{scope}[#1]
    \foreach \i in {1,2,...,#2} {
      \pgfmathsetmacro{\x}{\startangle + \angle*\i}
      \draw  (\x:2 cm) -- (\x + \angle:2 cm);
	\node (\i) at ( \x + \angle/2, 1.5) {1};
    }
  \end{scope}
}



\graphicspath{ {images/} }

\begin{document} 

\section{context}
\textbf{This is a first draft, context will be inserted}

\section{Result}


\begin{dfn}
Let $S$ be a cyclic quotient singularity, throughout the rest of this chapter we limit ourselves to cyclic quotient singularities $S$ with the following property: let $C_1, \dots ,C_n$ be the minimal resolution of $S$, let the values $a_1, \dots ,a_n$ be the respective discrepancies, we insist that $a_i \leq -\frac{1}{2}$. We call this a singularity with small discrepancy
\end{dfn}
This can be compared to other people's work, in the following lemma.
\begin{prop}
A singularity $S$ has small discrepancy if and only if $C_1^2 \neq 2$ and $C_n^2 \neq 2$.
\end{prop}
\begin{proof}
 We use the fact that the discrepancy is a strictly decreasing sequence then a strictly increasing sequence. So it suffices to show this for $C_1$ and $C_n$. We only care about the case of where the square is $3$. Without loss of generality we can assume $d_1 \geq d_2$ so $ d_2 \leq -1 - \frac{2 + d_2}{-3}$ rearranges to $2d_2 + 1 < 0$. Substituting this back into the equation for $d_1$ we get $d_1 \leq  \frac{-1}{2}$. 
 \end{proof}
\begin{lem}
Let $X$ be a surface and  $f \colon : Y \rightarrow X$ be the minimal resolution of $X$. Let $C \subset X$ be a curve that such that $C$ intersects two singularities (potentially the same) with small discrepancy. Consider the curve $\widetilde C \subset Y$ the strict transform. Then if $\widetilde C^2 = -1$ then $-K_X \cdot C \leq 0$.
\end{lem}
\begin{proof}
Let $f \colon : Y \rightarrow X$ be the minimal resolution of $X$, $\widetilde C \subset Y$ the strict transform of $C$. As $\widetilde C$ is a smooth curve on a smooth surface $K_Y \cdot \widetilde C = -1$. We know that $C$ intersects at least two exceptional curves $E_i, \, E_j$, with discrepancy $a, \, b$. Hence we see that $K_X \cdot C = f^*(K_X) \cdot \widetilde C \geq -1 + a + b  \geq 0$. 
\end{proof}
Hence this curve configuration cannot lie on a \ldp. We also make the quick remark that in the case where the length, $n$, of the singularity is 1 or 2, this lemma follows via easy toric geometry as any curve joining two singularities is a locally toric configuration. This corresponds to the associated fan being non convex. 
\begin{lem}
Let $X$ be a \ldp with only singularities of small discrepancy. As above let $f \colon : Y \rightarrow X$ be the \minres then we consider the map $\pi \colon : Y \rightarrow \mathbb{F}_l$. Consider the curves $E_i^S \subset Y$ which are the excpetional curves arising from the resolution of a singularity $S$. Then $\pi_* E_i^S$ is a smooth rational curve with self intersection $-l, \,0, \, l, \, l+2$ or a point. In addition for all singularities $S$ in $X$, there exists a curve $E_j^S$ such that $\pi_* E_j^S$ is not a point.

\end{lem}
\begin{proof}
We first show that it is impossible for $\pi_* E$ to be a non smooth curve. Hence assume that it has a singular point $P$. In order for us to get a cyclic quotient singularity, $E$ needs to be smooth. Hence there is a collection of curves $C_i \subset Y$ which blowdown to $P$. As all these curves are contracted $C_i^2 \leq  -1$ and $E^2 \leq 0$, and there is a curve $C_j$ with $C_j^2 = -1$. Clearly $C_j$ intersects either two curves with self intersection less than $-1$, or it could intersect $E$ twice. By the above lemma neither case could appear on the \minres of a \ldp.
\\
\\
To show that not all the curves $E_j^S$ can be contracted to a point, we go for a proof by contradiction. Assume they are all contracted to a point $P \in \mb{F}_l$. Then $P$ lies on a fiber $F$ which intersects the curve $B$. We have to blow up $P$ to get hence we get the following curve configuration. 
\[
B, -1, -1
\]
This configuration is locally toric. To recover the curve $E_i^S$ we  start with the toric blowups. Assuming we have done at least one blowup this results in a curve configuration with at least one $-1$ curve joining together two curves with self intersection less than $-1$. If this curve stayed a $-1$ curve after the non toric blowups then we would have a $-1$ curve joining together two singularities. This contradicts the previous lemma. Hence the $-1$ curve has to be blown up in a non toric way. This would result in it being a $-a$ curve with $a>1$. Hence it is exceptional and contracted. This results in our curves all being connected to via a chain of curves  with self intersection less than $-1$ to $B$. This means there exists $i$ such that $B = E_i^S$ contradicting our assumptions.
\end{proof}
Now we can classify these log del Pezzos in a straightforwards way. 
\begin{thm}
Let $X$ be a \ldp with only singularities of small discrepancy. Then $X$ has either one singularity or two $\frac{1}{p}(1,1)$ singularities or it does not occur. If $X$ admits no floating $-1$ curves then $X$ admits a toric degeneration.
\end{thm}
\begin{proof}
Given a \ldp $X_0$ we start by contracting all floating $-1$ curves. This gives rise to a \ldp $X_1$, let $Y$ be the minimal resolution of $X_1$. We know that there is a map $\pi : \colon Y \rightarrow \mathbb{F}_l$. We start by considering the case $l > 1$. There is a curve $B \subset \mathbb{F}_l$ with $B^2 = -l$. Assume there is no  $l' >l$ such that $Y \rightarrow \mb{F}_{l'}$. Then $B$ has to be the image of an exceptional curve $E_i$ inside $Y$. Assume our map $\pi$ involves blowing up a point on $B$. Without loss of generality we can assume that this is the first blowup, so we have curves $C_1, \, C_2$ which are both $-1$ curve, with $C_2$ being the strict transform of $0$ fiber. We could then instead contract $C_2$ and get a larger value of $l$. Hence this does not occur. 

We will deal with the case of $l = 0$ and $l = 1$ after this. Now there is a singularity $S$ such that $B \in \{ \pi_*E_i^S \}$. Assume $S$ is not a $\frac{1}{p}(1,1)$ singularity. Note that there is a curve $E_j^S$ such that $\pi_* E_j^S$ is a $B$, hence $E_{j\pm 1}^S$ is a $0$ curve or a $l+2$ curve, as we are assuming $l$ is the largest possible value of $l$ and hence $B$ could not be blown up. Denote these two curves by $C_1$ and $C_2$. Assume there was another singularity giving rise to exceptional curves $\{ E_i^{S'} \}_{0}^{m_{S'}} $. Then by the previous lemma there would be a curve $E_j^{S'}$ such that $\pi_* E_j^{S'}$ is a curve with self intersection $0, \,  l,\,  l+2$. However these curves would intersect $C_1$ and $C_2$ contradicting this being a distinct singularity. Hence there is only one singularity. If $S$ is a $\frac{1}{p}(1,1)$ singularity. Then consider the possibility of there being another singularity $S'$. There is a curve $E_j^{S'}$ such that $\pi_* E_j^{S'}$ has self intersection $l$, not $0$ or $l+2$ too avoid it meeting $B$. Denote this image by $A$. If $S'$ is not a $\frac{1}{p}(1,1)$ then we would have to blow up this curve several times. However each blowup introduces a $-1$ which is joined to curve $B$ by another $-1$ curve. Hence this curve cannot be blown up further, by the argument in the above lemma. If there was a third $\frac{1}{p}(1,1)$ singularity, then its exceptional curve would have to be sent to a $0, \, l, \, l+2$, however all of these would intersect the curve $A$ or intersect the curves introduced by blowing up points on $A$. In the first case it contradicts it being a new singularity and in the second it contradicts the singularities not being joined by a $-1$ curve.
\\
\\
Dealing with the case of $l = 0$ first. Assume that $Y$ is such that $Y$ only admits a map down to $\mb{F}_0$. However a blow up of any point of $\mb{F}_0$ results in a map to $\mb{F}_1$. So the only possibility is $\mb{F}_0$ itself. For $\mb{F}_1$ other cases arise. Clearly if we blow up a point on the $-1$ curve we get a map to $\mb{F}_2$. So the only option is a blowup at a smooth point. This results in three adjacent $-1$ curves. If we blowup a point on either of the two end curves we could get a map to $\mb{F}_2$. So the only option is blowing up the middle curve arbitrarily many times. This results in an infinite family of \ldp's with a single $\frac{1}{p}(1,1)$ singularity. 
\\
\\
We finish by discussing the condition that there are no floating minus one curves. We note that in the case where there is a curve $E_i^S$ such that $\pi_* E_i^S$ is an $l+2$ curve then the blowup introduces floating $-1$ curves corresponding to the $l$ curve that goes through $l+1$ of the points blown up. Hence this surface is not minimal.
\end{proof}


 This leads to the following corollary.
\begin{cor}
Let $X$ be a surface such that the basket is  $\{ \{ \frac{1}{p_1}(1,1), \, \dots \, \frac{1}{p_n}(1,1) \}, \, n \}$ , with the condition that either $p_i \geq 5$. Then there are at most two singularities. The case of one singularity was classified by CP. In the case of two singularities there is no cascade. 
\end{cor}
\begin{proof}
With these constrictions on singularities, it fits the criterion for the above theorem. The explicit classification was done in the case fo the theorem. We note that the cascade is of length 3 and all the surfaces admit toric degenerations. Let $X$ be the surface with no floating curves. Then $X$ admits a toric degeneration to $(-p_1, -1), \, (0, 1), \, (p_2, 1)$. Here the smoothing is equivariant with respect to the torus action. We have $-K_X^2 = \frac{4}{p_1} + \frac{4}{p_2}$. Even in cases where $-K_X^2 > 1$ we see that $X$ cannot be blown up while preserving $-K_X$ ample. If $X$ admitted a blow up at a general point $P$ then there is a fiber $F$ such that $P \in F$. Then $\widetilde F$ is a $-1$ curve on the minimal resolution connecting the $-p_1$ curve with the $-p_2$ curve. This is a contradiction.
\\
\\
We note that the total family can be see as a hypersurface of degree $p+q$ inside $\mb{P}(1,\,1,\,p,\,q)$.
\end{proof}
We now do a more difficult example by classifying the \ldp's with singularities $S_{a,b}$ with resolution $E_1, \, E_2$ with $E_1^2 = -a,\, E_2^2 = -b$. To make sure that this obeys they conditions on the theorem we insist $a, \, b \neq 2$. We note that the case of $S_{3,3}$ does not satisfy the conditions for the theorem. However we are interested in $\mb{Q}$g smoothings and $S_{3,3}$ is not $\mb{Q}$g rigid and admits admits a partial smoothing to $\frac{1}{6}(1,1)$ singularity. These were classified above. This is a more complicated example of how the above theorem can be used.
\\
\\
There are two heads of the cascade given by the following four surfaces. These correspond to surfaces constructed by blowing up $\mb{F}_a$ in $b$ points and $\mb{F}_b$ in $a$ points, then contracting the negative curves. We call these surfaces $X_a$ and $X_b$ respectively. These admits a toric degeneration to $\mb{P}(1, b, ab-1)$ and $\mb{P}(1, a, ab-1)$ respectively. We only consider the case of $X_a$ as $X_b$ is completely symmetric. We see that the we can smooth it by taking the $b$'th veronese embedding and  getting $\mb{P}_{u,\,v,\,w,\,t}(1, 1, ab-1, a)$ with the relation $uw = t^b$. This admits a smoothing giving us the surface lies as $X_{ab} \subset \mb{P}(1,\, 1, \, ab-1, \, a)$. This gives us $-K_X^2 = \frac{a^2(b+1)^2}{b(ab-1)}$. We note that this admits a cascade of length $a+2$. The first $a$ terms are easy to describe as we see that these admit a toric degeneration to $X_\Sigma$ with $\Sigma$ being the fan with rays $(-1, b), \, (-1, 0), \, (a, -1), \, (a-u, -1)$, where $u$ is the number of blowups. The $(a+1)$'st blowup admits a toric degeneration to $(-1, b), \, (-1, -1), \, (a, -1)$. The toric degeneration of the $a+2$'nd blowup is a bit more finicky and goes on a case by case analysis. However we see it admits a degeneration as it is a \LJ. We note that there is a birational relationship between $X_a$ and $X_b$ in that the $a$'th blowup of $X_a$ is isomorphic to the $b$'th blowup of $X_b$.  This generalises, in the following way. 
\begin{thm}
Given a singularity with small discrepancy such that the minimal resolution is $a_1, \dots a_n$, this has at most $n$ basic surfaces form the previous theorem, possibly less via symmetry. Let $X_i$ be the surface constructed from $X_{a_i}$, if $i \neq 1,\, n$ then this admits a cascade of length $a_i + 3 - n$, if $i = 1, \, n$ then the cascade is of length $a_i + 4 - n$.  After $a_i$ blowups $X_{i}$ is isomorphic to $X_j$ blownup $a_j$ times.
\end{thm}
\begin{proof}
Assume the length of the singularity is not one. We start by noting that $Y_i$ the minimal resolution of $X_i$ a cycle $-K_{Y_i}$ with intersections $-a_1, \dots -a_n, a_i+4-n$. If $a_i+4 -n \leq -2$ then this is clearly not the minimal resolution of a \ldp with cyclic quotients. If $a_i+4-n = -1$ then we have a $-1$ curve intersecting the singularity twice. 
\end{proof}
\begin{cor}
Let $S$ be a singularity with small discrepancy, $-a_1, \dots , -a_n$ be the self intersection of the resolutions. Then if $n \geq \max (a_i) + 5 $. Then there exists no \ldp with only singularities of type $S$.
\end{cor}

\end{document}