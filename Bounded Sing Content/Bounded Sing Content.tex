\documentclass[11pt]{amsbook}

\usepackage{geometry}  
\geometry{letterpaper} 
\usepackage{graphicx}
%\usepackage[backend=bibtex]{biblatex}
\usepackage{array}
\usepackage{amssymb}
\usepackage{amsmath}
\usepackage{amsthm}
\usepackage{graphicx}
\usepackage[parfill]{parskip} 
\usepackage[utf8]{inputenc}
\usepackage[english]{babel}
\usepackage{enumerate}
\usepackage{tikz}
\usepackage{tikz-cd}
\usepackage[noend]{algpseudocode}
\usepackage{caption}
\usepackage{subcaption}
\usepackage{fancyhdr}
\usepackage{enumitem}
\usepackage[super]{nth}
\usepackage{pstricks}
\usepackage{xstring}
\usepackage{comment}
\usepackage{pgfplots}
\usepackage{blkarray} %This is for labelled matrices
\usetikzlibrary{decorations.pathreplacing} % this is for Ai's drawing


\usepackage[colorlinks=true,linkcolor=blue]{hyperref}


\pagestyle{fancy}
\lhead{}
\chead{}
\rhead{}
\lfoot{}
\cfoot{\thepage}
\rfoot{}
\renewcommand{\headrulewidth}{0pt}
\setlength{\footskip}{50pt}

\makeatletter
\def\BState{\State\hskip-\ALG@thistlm}
\makeatother

\theoremstyle{definition}
\newtheorem{thm}{Theorem}[section]
\theoremstyle{definition}
\newtheorem{cor}[thm]{Corollary}
\theoremstyle{definition}
\newtheorem{prop}[thm]{Proposition}
\theoremstyle{definition}
\newtheorem{dfn}[thm]{Definition}
\theoremstyle{definition}
\newtheorem{lem}[thm]{Lemma}
\theoremstyle{definition}
\newtheorem{ex}[thm]{Example}
\theoremstyle{definition}
\newtheorem{conj}[thm]{Conjecture}
\theoremstyle{definition}
\newtheorem*{rem}{Remark}
\newtheorem{assumption}[thm]{Assumption}

\newcommand{\Rom}[1]
    {\MakeUppercase{\romannumeral #1}}
\newcommand{\C}[1]{(\mathbb{C}^*)^#1}
\newcommand{\ldp}{log del Pezzo}
\newcommand{\mb}[1]{\mathbb{#1}}
\newcommand{\Hi}{Hirzebruch surface }
\newcommand{\minres}{minimal resolution}
\newcommand{\LJ}{Looijenga pair}
\newcommand{\ra}{\rightarrow}
\newcommand{\spl}{\text{SL}_2 (\mathbb{C})}
\newcommand{\gl}{\text{GL}_2 (\mathbb{C})}
\newcommand{\pgl}{\text{PGL}_2 (\mathbb{C})}
\newcommand{\wt}[1]{\widetilde #1}
\newcommand{\Q}{\mathrm{Q}}
\newcommand{\Z}{\mathrm{Z}}
\newcommand{\F}{\mathrm{F}}
\renewcommand{\P}{\mathrm{P}}






\graphicspath{ {images/} }

\begin{document} 

\setcounter{chapter}{2}

\section{context}

\textbf{This is a first draft, context will be inserted}

\section{Standard notions and notation for quotient singularities}
\label{sec!notation}

\section{Log del Pezzo surfaces with small discrepancy}

Recall from Section~\ref{sec!notation} our standard notation for quotient singularities.
We consider the germ $S$ of a cyclic quotient singularity appearing at a point $P$ on a 
projective surface $X$.
The minimal resolution of $X$ is denoted $f\colon Y \longrightarrow X$. It contains a chain of
exceptional (smooth, rational)
curves $C_1,\dots,C_n$, entirely determined by $S$ itself, which are ordered so
that the only intersections between these curves are
$C_i\cap C_{i+1}$ which is a single transverse intersection for each $i=1,\dots,n-1$; 
in other words,
$C_1$ and $C_n$ are the two `ends' of the chain.
We also denote the discrepancies of each $C_i$ (as curves in $Y$) by $d_i\in\Q$: thus
\[
K_Y = f^*(K_X) + \sum_{i=1}^n d_i C_i.
\]
We introduce a property of cyclic quotient singularities that is central to the rest of the chapter.
\begin{dfn}
Let $S$ be a cyclic quotient singularity, and $C_1, \dots ,C_n$ the exceptional curves of the minimal resolution of $S$ and $d_1, \dots,d_n$ their discrepancies, as above.
We say that $S$ is a \emph{ singularity with small discrepancy} if $d_i \leq -\frac{1}{2}$ for
all $i=1,\dots,n$.
\end{dfn}

\begin{prop}
In the notation above,
a singularity $S$ has small discrepancy if and only if $C_1^2 \neq -2$ and $C_n^2 \neq -2$.
\end{prop}
\begin{proof}
 We use the fact that the discrepancy is a strictly decreasing sequence then a strictly increasing sequence. So it suffices to show this for $C_1$ and $C_n$. We only care about the case of where the square is $3$. Without loss of generality we can assume $d_1 \geq d_2$ so $ d_2 \leq -1 - \frac{2 + d_2}{-3}$ rearranges to $2d_2 + 1 < 0$. Substituting this back into the equation for $d_1$ we get $d_1 \leq  \frac{-1}{2}$. 
 \end{proof}


Throughout the rest of this chapter we restrict the class of singularities we consider as follows:

\begin{assumption}
Any singularity germ $S$ that appears in this chapter is assumed to be a cyclic
quotient singularity with small discrepancy.
\end{assumption}

\begin{lem}\label{lem!badcurve}
Let $X$ be a surface having cyclic quotient singularities of small discrepancy, and let  $f \colon Y \rightarrow X$ be the minimal resolution of $X$. Let $C \subset X$ be a rational curve whose 
strict transform $\widetilde C \subset Y$ is smooth. Suppose in addition that 
$\widetilde C$ meets the exceptional locus of $f$ with intersection multiplicity at least~2.
Then if $\widetilde C^2 = -1$ then $-K_X \cdot C \leq 0$.

In particular, $\widetilde C$ is smooth and
$C$ either meets at least two singularities of $X$ or meets one singularity
with at least branches or has a singular point of $C$ at a singularity of $X$,
then the hypotheses on $C$ are satisfied.
\end{lem}
\begin{proof}
%Let $f \colon : Y \rightarrow X$ be the minimal resolution of $X$, $\widetilde C \subset Y$ the strict transform of $C$. 
By the genus formula for $\widetilde C\subset Y$, as $\widetilde C$ and $Y$ are both smooth,
$K_Y \cdot \widetilde C = -1$. If $\wt C$ intersects two distinct exceptional curves $E_i$, $E_j$,
with discrepancy $d_i$, $d_j$ respectively, then
 $K_X \cdot C = f^*(K_X) \cdot \widetilde C \geq -1 - d_i - d_j  \geq 0$,
 as $X$ has only singularities with small discrepancy. 
 If, on the other hand, $\wt C$ meets only one exceptional curve $E_i$, but with intersection
multiplicity $m_i$, then $K_X \cdot C = f^*(K_X) \cdot \widetilde C \geq -1 - m_id_i  \geq 0$.
\end{proof}

We show next that in fact such rational curves cannot lie on a \ldp.
We need a preliminary lemma.
\begin{lem}\label{lem!minus2curve}
Let $X$ be a \ldp\ and $f \colon Y \rightarrow X$ be the \minres.
Let $C\subset Y$ be a smooth rational curve. If $C^2\le-2$ then $C$ is contracted by $f$
to a point of~$X$.
\end{lem}

\begin{proof}
We proof this by contradiction. Assume there is a curve $C$ that is not contracted, the $K_X \cdot f(C) = f^*(K_X) \cdot \wt{C} \geq K_Y \cdot \wt{C} \geq 0$, with the inequality following as there are no terminal surface singularities.
\end{proof}

\begin{prop}\label{MainProp}
Let $X$ is a \ldp\ with singularities of small discrepancy and 
consider the following diagram
\[
% https://tikzcd.yichuanshen.de/#N4Igdg9gJgpgziAXAbVABwnAlgFyxMJZARgBoAGAXVJADcBDAGwFcYkQANEAX1PU1z5CKAEyli1Ok1bsAmjz4gM2PASLlxkhizaIQALR6SYUAObwioAGYAnCAFskGkDghIyUnewA63uAGMbLDQcOBwAT0YYYFNuEBpGegAjGEYABQFVYRAg0wALHAVrO0dEZ1ckMU8ZPV8AoJCwyOirOO5KbiA
\begin{tikzcd}
  & X \arrow[rd, "\scriptstyle{g}"'] \arrow[ld, "\scriptstyle{f}"] &   \\
Z &                                                                & Y
\end{tikzcd}
\]
$f$ is the minimal resolution of $X$ and $g$ is a birational
morphism to a smooth surface $Z$.
Let $E\subset Y$ be an $f$-exceptional curve. Then $E$ is contracted to a point
of $Z$ by $g$, or $g(E)$ is a smooth curve and $g_E$ is an isomorphism.
\end{prop}

\begin{proof}
Let $E\subset Y$ be any one of the exceptional curves $E_i^S$; in particular, $E$ is
a smooth rational curve with $E^2 \le-2$.
We first show that if $f_* E\subset Z$ is a curve, then it must be a smooth curve. 

For contradiction, suppose $f_*E$ is a curve with a singular point $P$.
Let $C_1,\dots,C_s\subset Y$ be the curves that contract to $P$ under~$f$.
As these curves are contracted, $C_i^2 \leq  -1$.
Notice that if $C_i^2\le-2$, then $f(C_i)$ is a point of~$X$
by Lemma~\ref{lem!minus2curve}.
There are two cases to consider: set-theoretically, either
$\pi^{-1}(P)$ meets $E$ in a single point or in more than one point.


In the case of more than one intersection point, since $\pi^{-1}(\pi_*(E))$ is connected,
among the curves $C_i$ there must be a shortest chain $C_1\cup\cdots\cup C_r$
with $C_k\cdot E=0$ for $k=2,\dots,r-1$, and $\left(\sum_{i=1}^r C_i\right)\cdot E = 2$.
At least one of the curves $A=C_k$ of the cycle must have $A^2=-1$, otherwise the
whole cycle is contracted to a point $R$ of $X$, but then $R\in X$ would not be
a rational singularity, and so in particular not a cyclic quotient singularity.
And of course $A$ cannot meet another $-1$-curve $C_j$ with $\pi(C_j)=P$.
Thus $A$ must lie in one of the following configurations:
\begin{enumerate}
\item
$A$ meets two distinct $\pi$-exceptional curves, $C_j$ and $C_{j'}$,
both of which have self-intersection $\le-2$.
\item
$A$ meets $E$ in one point and a distinct $\pi$-exceptional curves $C_j$
with $C_j^2\le-2$.
\item
$A$ meets $E$ in two distinct points.
\end{enumerate}
In each of these situations, $C = f_*(A)\subset X$ would be a curve
on which $K_X$ is nef, by Lemma~\ref{lem!badcurve}, which contracts
$X$ being \ldp. Indeed $A = \wt C$ meets the $f$-exceptional locus with multiplicity
at least~2 in each case.

The argument in the nodal case follows similarly, up to the case division of configurations
at which there is an additional case:

We note that if $\pi^{-1}{P}$ contains two curve $C_i$, $C_j$ with $I_Q (C_i, \, C_j) \ge 2$ then either one of them is a $-1$-curve or it cannot occur on the minimal resolution of a \ldp\ surface. This is because every curve in $\pi^{-1}{P}$ has negative self intersection, if its intersection is less than $-1$ then it would have to be contracted on the map down to $X$, resulting in a non cyclic quotient singularity. Hence one of them is $-1$ curve, and this cannot occur as it would contradict Lemma~\ref{lem!badcurve}. Hence this cannot occur, so we have to blow up the point $P$ enough times such that all the intersections are transverse. At this point we have a curve $A$ such that $A$ intersects transversely at least three other curves, $E, \, C_1, \, C_2 \dots $ with $C_i \in \pi^{-1} {P}$. In addition $C$ is the only $-1$ curve in $\pi^{-1}{P}$. As $Y$ is constrcuted from further blowups we split into configurations 

\begin{enumerate}
\item
The strict transform of $A$, denoted $\widetilde{A}$ has $\wt{A}^2 = -1$.
\item
The strict transform of $A$, denoted $\widetilde{A}$ has $\wt{A}^2 \leq -2$.
\end{enumerate}

In the first case Lemma~\ref{lem!badcurve} this cannot occur on the minimal resolution of a \ldp\ surface due to the curves $\wt{E}, \, \wt{C_1},\, \wt{C_2}$. In the second case, if none of the intersection points $A\cap E, \, A \cap C_1, \, A \cap C_2$ have been blown up then we are left with a non cyclic quotient singularity. Hence one of these points has to be blown up. This results in a $-1$-curve intersecting $\wt{A}$ and another negative curve hence we have a contradiction to Lemma~\ref{lem!badcurve}.


For a completely general curve singularity it follows by a combination of the above arguments. 
\end{proof}

\begin{lem}\label{HSlem}
Let $X$ be a \ldp\ with only singularities of small discrepancy, and
let $f \colon Y \rightarrow X$ be the \minres. We suppose $X\not=\P^2$, so that $\rho_Y\ge2$.
The resolution $Y$ admits a morphism $\pi \colon Y \rightarrow \mathbb{F}_l$ for some $l\ge0$:
this is the minimal model of~$Y$, unless that minimal model
is $\P^2$, in which case there is a factorisation
$Y\rightarrow \F_1\rightarrow\P^2$.

For a germ $S$ of a singularity of~$X$, denote by
$E_i^S \subset Y$ the exceptional curves in the resolution of $S$.
For each singularity $S$ on $X$:
\begin{enumerate}
\item
Every exceptional curve $E_i^S$ is either contracted to a point of $\mb{F}_l$ by $\pi$,
or the pushdown
$\pi_* E_i^S\subset\F_l$ is a smooth rational curve with self intersection one of $-l, \,0, \, l, \, l+2, \, l+4,\ 4l $, where $l+4$ can only occur for $l<2$.
\item
In the case $l\ge2$, there is always some curve $E_j^S$ not contracted by $\pi$.
\end{enumerate}

\end{lem}
\begin{proof}
To prove the first statement note that $\pi_* E_i^S$ cannot be a singular curve by Proposition~\ref{MainProp}, hence it is a smooth rational curve. The only smooth rational curves on a Hirzebruch Surface $\mb{F}_l$ are the curves $B$, with $B^2 = -l$, $F$ with $F^2 = 0$ and the curves lieing inside the linear systems $|lF + B|$,  $|(l+1)F + B|$, $|2F|$, $|2(lF+B)|$  and finally $|(l+2)F + B|$. We note that the final case could not arise on $\mb{F}_l$ when $l \ge 2$.  In this case the curve $B$ is also the image of an exceptional curve from a singularity. Hence any curve in $|(l+2)F + B$ would intersect $B$, when counting multiplicities, $2$ times. This would be a contradiction to Lemma~\ref{lem!badcurve}. A similar argument occurs with $2F$ which is meeting the curve $B$ at a single point with multiplicity 2.



To show that not all the curves $E_j^S$ can be contracted to a point if $l \geq 2$, we go for a proof by contradiction. Assume $l \ge 2$ and every exceptional curve in a singularity $S$ is contracted to a point $P \in \mb{F}_l$. Then $P$ lies on a fiber $F$ which intersects the curve $B$. First we consider $P \not\in B$. We have $E_i^S \in \pi^{-1}{P}$ for all $i$. Hence we have to blow up several times. However the strict transform of the fiber $F$, denoted $\wt{F}$ now has $\wt{F}^2 \leq -1$. If $\wt{F}^2 \leq -2$ then it has to be contracted, meaning $\wt{F}, \, B \in \{ E_i^S \}$ which would be curves not contracted to a point. If $\wt{F}^2 = -1$, then the only $-1 $ curves in $\pi^{-1}{P}$ cannot intersect $\wt{F}$. This is because after the first blowup we have an exceptional curve $E$ and the fiber $\wt{F}$. These both have square $-1$. If we blow up the intersection point of $\wt{F}$ and $E$ then $\wt{F}^2 \leq -2$, hence we can only blowup general points on $E$. At this point we have non e of the $-1$-curves intersecting $E$. If we blowup no points on $E$ then clearly we are not introducing a singularity so this does not occur. Now finally we note that our curve configuration would contradict Lemma~\ref{lem!badcurve}. 

\end{proof}

\begin{rem}
In the case where the length, $n$, of the singularity is 1 or 2, Lemma~\ref{lem!badcurve} follows via easy toric geometry as any curve joining two singularities is a locally toric configuration. This corresponds to the associated fan being non convex. 
\end{rem}

Now we can classify these log del Pezzos in a straightforwards way. 
\begin{thm}\label{ThmOnSing}
Let $X$ be a non-smooth \ldp\ with only singularities of small discrepancy. Then 
\begin{enumerate}
\item\label{thm38i}
$X$ has either one singularity or two singularities, each of type $\frac{1}p(1,1)$ for some, possibly different,~$p$.
\item\label{thm38ii}
If $X$ admits no floating $-1$-curves then $X$ admits a toric degeneration. %In particular given a singularity $S$ we have at most $m$ basic surfaces, where $m$ is the number of exceptional curves in the resolution of $S$.
\end{enumerate}
\end{thm}
\begin{proof}
Given a \ldp\ $X_0$ we start by contracting all floating $-1$ curves. This gives rise to a \ldp\ $X_1$; note that $X_1$ is not $\P^2$ since the contraction map is an isomorphism in the neighbourhood of any singularity of $X_0$. Let $\sigma\colon Y\rightarrow X_1$ be the minimal resolution of $X_1$. We know that there is a map $\pi \colon Y \rightarrow \mathbb{F}_l$, and we may suppose $l$ is maximal.
%We start by considering the case $l > 1$. 
There is a curve $B \subset \mathbb{F}_l$ with $B^2 = -l$. 
%Assume there is no  $l' >l$ such that $Y \rightarrow \mb{F}_{l'}$. 
If $l\ge2$ then $B$ has to be the image of a $\sigma$-exceptional curve $E_i$ inside $Y$.

We first show that $\pi$ cannot contract a curve to a point on $B$.
If on the contrary there is a curve contracted to $B$, then
without loss of generality we may assume that it is the exceptional curve of the final blowdown $Y\rightarrow Y_2\rightarrow\F_l$. In that case, there two curves $C_1, \, C_2$ on $Y_2$, both $-1$ curves, with $C_2$ being the strict transform of $0$ fiber. But then we could instead contract $C_2$ from $Y_2$ and get a map to $\mb{F}_{l+1}$, contradicting maximality of~$l$. 
Hence $\pi$ is indeed an isomorphism in a neighbourhood of~$B$.


We first consider $l\geq 2$. Now there is a singularity $S$ such that $B \in \{ \pi_*E_i^S \}$. Assume first that $S$ is not a $\frac{1}{p}(1,1)$ singularity. Note that there is a curve $E_j^S$ such that $\pi_* E_j^S$ is $B$. The adjacent (one or two) exceptional curves cannot be contracted (by the argument of the previous paragraph). We suppose there are two adjacent curves $E_{j\pm 1}^S$; the case where $E_j^S$ is at the end of a chain of blowups with only one adjacent exceptional curve works in exactly the same way. Thus each of $\pi_*E_{j\pm 1}^S$ is either a $0$ curve (a fiber) or an $l+2$ curve on~$\F_l$ (by the classification of smooth rational curves on $\F_l$ in Lemma~\ref{introlemmaonFl}).
%, as we are assuming $l$ is the largest possible value of $l$ and hence $B$ could not be blown up. 
Denote these two adjacent curves by $C_1$ and $C_2$ respectively. Assume there was another singularity with exceptional curves $\{ E_i^{S'} \}_{i=0}^{m_{S'}} $ on $Y$. Then by Lemma~\ref{HSlem} there would be a curve $E_j^{S'}$ such that $\pi_* E_j^{S'}$ is a curve with self intersection $0, \,  l,\,  l+2$. However these curves would necessarily intersect $C_1$ and $C_2$ meaning either $S'$ is not distinct from $S$ or there is a $-1$-curve in $Y$ connecting two of their curves in the minimal resolution. Hence $X$ has precisely one singularity. 

To complete the analysis of this step, suppose $S$ is a $\frac{1}{p}(1,1)$ singularity and that its unique exceptional curve is mapped to the negative section~$B$. Then consider the possibility of there being another singularity $S'$ on~$X$. By Lemma~\ref{HSlem}, there is a curve $E_j^{S'}$ such that $A=\pi_* E_j^{S'}$ has self intersection $l$ or $4l$; it cannot be $0$ or $l+2$ as it must not meet~$B$. If $S'$ is not a $\frac{1}{p}(1,1)$ then there is at least one exceptional curve among the $E_k^{S'}$ that is contracted to a point on $A\subset\F_l$. However each blowup of a point $Q\in A$ introduces a $-1$-curve $D$ which is joined to curve $B$ by another $-1$-curve, the birational transform of the fiber through~$Q$. Hence none of these curves $E_k^{S'}$ can be mapped to $D$, as otherwise it would be adjoined to $B$ by a $-1$-curve, contradicting Lemma~\ref{lem!badcurve}.
Thus any other singularity on $X$ is also of type $\frac{1}{p}(1,1)$ (though possibly for a different~$p$).

Suppose now that there was a third singularity of type $\frac{1}{p}(1,1)$. Once again, its exceptional curve would have to be sent to a $0, \, l, \, l+2$. Any smooth rational curve on $\F_l$ with one of these intersection numbers intersects the curve $A$. Thus on $Y$ it must either meet the birational transform of $A$ or meet some curve that contacts to~$A$. Once again in the second case it will result in two singularities connected by a $-1$-curve. This is a contradiction to small discrepancy.
% or at least one of the curves that contr blowing up points on $A$. In the first case it contradicts it being a new singularity and in the second it contradicts the singularities not being joined by a $-1$-curve. 

Thus $X$ has exactly one or two singularities of type $\frac{1}p(1,1)$,
and part~\eqref{thm38i} is complete in the case $l\ge2$.


We now deal with the two cases we did not consider earlier $l = 0$ and $l =1$. Dealing with the case of $l = 0$ first, we know that $Y$ has no map to $\F_l$ for $l\ge1$, by maximality of~$l$. Thus $Y$ must be $\F_0$, since a blow up of any point of $\mb{F}_0$ also permits a map to $\mb{F}_1$; but then $X=Y$ is smooth, contradicting the assumption. 

For $\mb{F}_1$ other cases arise. Clearly if we blow up a point on the $-1$-curve we get a map to $\mb{F}_2$. So the only option is a blowup at a smooth point. Log del Pezzo surfaces do indeed arise in this way; see \cite[Table ??]{CP}.
This results in three adjacent $-1$-curves. Blowing up any point on either of the two end curves reveals a map to $\mb{F}_2$, so the only option for further blowups is at points above points of the middle curve.
% If the point is not in general position then we get a map to $\mb{F}_2$.
 This results in an infinite family of \ldp's with a single $\frac{1}{p}(1,1)$ singularity. We note that if we blowup two general points on $\mb{F}_1$ we get a surface with the property that for any $-1$-curve there is a map to $\mb{F}_1$ which sends it to the $-1$-curve on $\mb{F}_1$. Hence every surface arising this way would of arisen from our earlier case analysis.


%Hence to each choice of curve $C$ in the minimal resolution with $C^2 = a$ we get a corresponding \ldp\ with a map down to $\mb{F}_a$. It is fully possible that some of these surfaces may not exist, or may not be the minimal resolution of a \ldp. It is also possible that if the singularity has some symmetry, then there may be isomorphisms between these surfaces.



For part~\eqref{thm38ii}, we first observe that neither of the adjacent curves $E_{j\pm1}^S$ can
map to an $l+2$ curve, since in that case $X$ will have a floating $-1$-curve. This is because $l+1$ points in general position on the $l+2$ curve can be cut out as the intersection of the $l+2$ curve with an $l$ curve.

%We finish by discussing the condition that there are no floating minus one curves. We note that in the case where there is a curve $E_i^S$ such that $\pi_* E_i^S$ is an $l+2$ of an $l$ curve then the blowup introduces floating $-1$ curves corresponding to the $l$ curve that goes through $l+1$ of the points blown up. Hence this surface is not minimal.

Because of this we see that the only possibilities for $\pi_*(E_{j\pm1}^S)$ are two different $0$ curves. 
(Again we suppose there are two adjacent curves; the case of one adjacent curve is the same.)
We can then proceed to construct the configuration of all exceptional curves inductively. This means that when a surface of this form is able to be constructed we can obtain it by doing two weighted blowups at a general point of a Hirzebruch surface and then doing a series of non toric blowups on the boundary. The following surface is one example, arising from blowing up two general points of a Hirzebruch surface with weight $(1,i)$ and $(1,n-i)$. 
\begin{figure}[ht]
\begin{center}
\begin{tikzpicture}
  % nodes
	\node (0) at (-4.5, -3.5) {};
	\node (1) at (4.5, -3.5) {};
	\node (2) at (-3, -4) {};
	\node (3) at (-6, 0.75) {};
	\node (4) at (3, -4) {};
	\node (5) at (6, 0.75) {};
	\node (6) at (-3.5, 3) {};
	\node (7) at (3.5, 3) {};
	\node (8) at (-6, -0.75) {};
	\node (9) at (6, -0.75) {};
	\node (10) at (-6, 5) {};
	\node (11) at (-3.5, 8.75) {};
	\node (12) at (3.5, 8.75) {};
	\node (13) at (6, 5) {};
	\node (14) at (-4, 8.5) {};
	\node (15) at (4, 8.5) {};
	\node [label={[blue]below:$a_{i} +3-n$}] (16) at (0, 7.4) {};
	\node [label={above:$a_0$}] (17) at (-5, 6.75) {};
	\node [label={above:$a_n$}] (18) at (5, 6.75) {};
	\node (20) at (-5, 3.75) {$\vdots$};
	\node (20a) at (5, 3.75) {$\vdots$};
	\node [label={below:$a_i$}] (21) at (0, -3.5) {};
	\node [label={above:$a_{i-2}$}] (22) at (-5, 1.5) {};
	\node [label={above:$a_{i+2}$}] (23) at (5, 1.5) {};
	\node [label={below:$a_{i-1}$}] (22a) at (-5, -1.75) {};
	\node [label={below:$a_{i+1}$}] (23a) at (5, -1.75) {};
	%% a_{i-1}
	\node (24) at (-5.5, -1) {};
	\node (25) at (-2.75, 0.75) {};
	\node (26) at (-5, -1.75) {};
	\node (27) at (-2.25, 0) {};
	\node (28) at (-4, -3.25) {};
	\node (29) at (-1.25, -1.5) {};
	\node [red] (30) at (-3, -1.5) {$\vdots$};
	%% a_{i+1}
	\node (31) at (5.5, -1) {};
	\node (32) at (2.75, 0.75) {};
	\node (33) at (5, -1.75) {};
	\node (34) at (2.25, 0) {};
	\node (35) at (4, -3.25) {};
	\node (36) at (1.25, -1.5) {};
	\node [red] (37) at (3, -1.5) {$\vdots$};
	%% a_0
	\node (41) at (-6, 5.75) {};
	\node (42) at (-3.25, 4) {};
	\node (43) at (-5.5, 6.5) {};
	\node (44) at (-2.75, 4.75) {};
	\node (45) at (-4.5, 8) {};
	\node (46) at (-1.75, 6.25) {};
	\node [red] (47) at (-3.5, 6.5) {$\vdots$};
	%% a_n
	\node (49) at (6, 5.75) {};
	\node (50) at (3.25, 4) {};
	\node (51) at (5.5, 6.5) {};
	\node (52) at (2.75, 4.75) {};
	\node (53) at (4.5, 8) {};
	\node (54) at (1.75, 6.25) {};
	\node [red] (55) at (3.5, 6.5) {$\vdots$};
	% edges
	\draw (0.center) to (1.center);
	\draw (6.center) to (8.center);
	\draw (3.center) to (2.center);
	\draw (7.center) to (9.center);
	\draw [in=60, out=-120] (5.center) to (4.center);
	\draw (11.center) to (10.center);
	\draw (12.center) to (13.center);
	\draw [bend right,blue] (14.center) to (15.center);
	%% a_{i-1}
	\draw [red] (24.center) to (25.center);
	\draw [red] (26.center) to (27.center);
	\draw [red] (28.center) to (29.center);
	%% a_{i+1}
	\draw [red] (31.center) to (32.center);
	\draw [red] (33.center) to (34.center);
	\draw [red] (35.center) to (36.center);
	%% a_0
	\draw [red] (41.center) to (42.center);
	\draw [red] (43.center) to (44.center);
	\draw [red] (45.center) to (46.center);
	%% a_n
	\draw [red] (49.center) to (50.center);
	\draw [red] (51.center) to (52.center);
	\draw [red] (53.center) to (54.center);
	% braces
	\draw [decorate,decoration={brace,amplitude=5pt},xshift=5pt,yshift=3pt,red] (-2.75, 0.75) -- (-1.25, -1.5) 
	node [red,midway,xshift=2.5em,yshift=1ex] {$a_{i-1}-1$};
	\draw [decorate,decoration={brace,amplitude=5pt},xshift=-5pt,yshift=3pt,red] (1.25, -1.5) -- (2.75, 0.75) 
	node [red,midway,xshift=-2.5em,yshift=1ex] {$a_{i+1}-1$};
	\draw [decorate,decoration={brace,amplitude=5pt},xshift=5pt,yshift=-3pt,red] (-1.75, 6.25) -- (-3.25, 4) 
	node [red,midway,xshift=2em,yshift=-1ex] {$a_0-1$};
	\draw [decorate,decoration={brace,amplitude=5pt},xshift=-5pt,yshift=-3pt,red] (3.25, 4) -- (1.75, 6.25) 
	node [red,midway,xshift=-2em,yshift=-1ex] {$a_n-1$};
\end{tikzpicture}
\end{center}
\caption{Example of surface.\label{fig!logdp}}
\end{figure}
This is the picture where the map to the Hirzebruch surface is an isomorphism on an exceptional curve $E_i$, where $1<i<n$. Here the red curve s indicate $-1$-curves and the blue curves indicate curves with positive self intersection. The blue curve has self intersection $a_i+ 3 - n$. This value is dependent on $n \ge 3$ and the map to the Hirzebruch surface being an isomorphism on a curve $E_i$ with $1<i<n$. This is because on the Hirzebruch surface $\mb{F}_{a_i}$ this curve had self intersection $a_i$. The $n-3$ of our exceptional curves are mapped to points and hence it has self intersection $a_i-(n-3)$. In the case that our map is an isomorphism on the curve $E_1$ or $E_n$ then we have a similar looking configuration except with positive curve now having self intersection $a_i + 4 - n$ as there is an extra point being blown up however the image of the curve in the Hirzebruch surface now is in the linear system $|(l+1)F + B|$. Hence its self intersection is now $l -(n - 3) -1 = l - n + 4$.

The toric degeneration property now follows. By construction all these surfaces $X$ are \LJ s,
and so admit a toric degeneration by \cite[Theorem ??]{GHK}.
%When these singularities are $\mb{Q}$-Gorenstein rigid we get a unique corresponding singularity content, as there is only one surface. 
\end{proof}


This leads to the following corollary in which we classify all log del Pezzo surfaces with singularities of small discrepancy, each of which is resolved by a single exceptional curve.
 
\begin{cor}
Let $X$ be a log del Pezzo surface with small discrepancy and
basket  $\{ \{ \frac{1}{p_1}(1,1), \, \dots \, \frac{1}{p_n}(1,1) \}, \, m \}$
for $n\ge0$ and $m\ge0$. Then $n\le2$ and moreover
\begin{enumerate}
\item\label{dp11i}
if $n\le1$ then either $X$ is a smooth del Pezzo surface or 
lies in a cascade over $\P(1,1,k)$ (see \cite[Table ??]{CP});
\item\label{dp11ii}
if $n=2$ then $X$
is isomorphic to a quasismooth weighted hypersurface
$X_{p+q}\subset \mb{P}(1,\,1,\,p,\,q)$. Conversely any such hypersurface with $p,q\ge4$ is
a log del Pezzo surface with small discrepancy.
\end{enumerate}
In particular, in the case of two singularities there is no cascade.
\end{cor}

The small discrepancy condition is equivalent to the condition that
$p_i \geq 4$ for each $i=1,\dots,n$. 
For the sake of completeness, we outline the classification result of \cite{CP} that
describes part~\ref{dp11i}, which also follows independently from MY PROOF ? LEMMA ??

\begin{proof}
With these constrictions on singularities, it fits the criterion for the above theorem. The explicit classification was done in the proof of the theorem. The case of one singularity was done in CP. The only examples of these surfaces with more than one singularity are constructed by blowing up a Hirzebruch surface in several points along a line and then contracting the two curves. Denote this surface by $X$. Then $X$ admits a toric degeneration to $(-p_1, -1), \, (0, 1), \, (p_2, 1)$. Note $X$ admits a $\mb{C}^*$ action and the degeneration is equivariant with respect to the torus action. We have $-K_X^2 = \frac{4}{p_1} + \frac{4}{p_2}$. Even in cases where $-K_X^2 > 1$ we see that $X$ cannot be blown up while preserving $-K_X$ ample. If $X$ admitted a blow up at a general point $P$ then there is a fiber $F$ such that $P \in F$. Then $\widetilde F$ is a $-1$-curve on the minimal resolution connecting the $-p_1$ curve with the $-p_2$ curve. This is a contradiction. Hence there is only one element in the cascade.



We note that this surface can be see as a hypersurface of degree $p+q$ inside $\mb{P}(1,\,1,\,p,\,q)$.

\end{proof}
We now do a more difficult example by classifying the \ldp's with singularities $S_{a,b}$ with resolution $E_1, \, E_2$ with $E_1^2 = -a,\, E_2^2 = -b$. To make sure that this obeys they conditions on the theorem we insist $a, \, b \neq 2$. We note that the case of $S_{3,3}$ does satisfy the conditions for the theorem. However we are interested in $\mb{Q}$g smoothings and $S_{3,3}$ is not $\mb{Q}$g rigid and admits a partial smoothing to $\frac{1}{6}(1,1)$ singularity. These were classified above. This is the only one of these singularities which is not $\mb{Q}$Gorenstein rigid. This is a more complicated example of how the above theorem can be used.


\begin{cor}\label{doublecurve}
Let $X$ be a surface such that the basket is  $(\{ S_{a_1, \, b_1}, \dots , S_{a_m, \, b_m} \}, \, n )$ , with the condition that $a_i, \, b_i \geq 3$ and we exclude the case $a_i = b_i = 3$. Then there is at most one singularities. They fall into a cascade of the form
\[
% https://tikzcd.yichuanshen.de/#N4Igdg9gJgpgziAXAbVABwnAlgFyxMJZABgBpiBdUkANwEMAbAVxiRAA0B9OgPWJAC+pdJlz5CKAIzkqtRizZdekwcJAZseAkQBMM6vWatEIADqmAxlAg4EQkZvFEAzPrlHF3HsDoBaSQIABKoOYtooACykkrKGCiZcKvbqoloSyACs0bHyxhycOiEpjuEkpDo5HgmcAEZ8RRph6dIVBrmedUlqjWm65ZXxZpbWtg2pTiiure6DXHXANf4CgrIwUADm8ESgAGYAThAAtkjSIDgQSGQzeeZoABZYXvzUDHQ1MAwACuPhIAwwOxwRX2RyQejOF0QpziN1M90eyhALzeH2+JQkfwBQOSIOOiFcELBbSqQ3hXh8vh0y2R7y+Pwx-0BwIOeIA7NRzkgAGzEwa3B61eo01H0th7LDrO7YtS4pAADg5kPZ1zY-MenSRfxRdPRYolUuZoMQAE5FfLebCyfNFlTNa9aWimnrJdLdiykFFCSaLaq4QLrUtDXjPZz8T6TGryX4AkGkFkvZ6Yb7Pg87drHb0TOKXSsBEA
\begin{tikzcd}
X_a^0 & X_a^1 \arrow[l, "\phi_a^0"]  & \cdots \arrow[l, "\phi_a^1"]  & X_a^{a-1}  \arrow[l, "\phi_a^{a-2}"] &                                                           &                        \\
      &                              &                               &                                      & X_1 \arrow[ld, "\phi_b^{b-1}"] \arrow[lu, "\phi_a^{a-1}"] & X_2 \arrow[l, "\Phi_1"'] & X_3 \arrow[l, "\Phi_2"']\\
X_b^0 & X_b^1 \arrow[l, "\phi_b^0"'] & \cdots \arrow[l, "\phi_b^1"'] & X_b^{b-1} \arrow[l, "\phi_b^{b-2}"'] &                                                           &                       
\end{tikzcd}
\]
\end{cor}

\begin{proof}

Once again by Theorem~\ref{ThmOnSing} there are two heads of the cascade given by the following two surfaces. These correspond to surfaces constructed by blowing up $\mb{F}_a$ in $b$ points and $\mb{F}_b$ in $a$ points, then contracting the negative curves. We call these surfaces $X_a$ and $X_b$ respectively.

\[
\resizebox{7cm}{5cm}{
\begin{tikzpicture}{s}
    \node (0) at (-2.5, 4) {};
	\node (1) at (-2.5, -4) {};
	\node (2) at (2.5, 4) {};
	\node [label={below:$-b$}] (3) at (2.5, -4) {};
	\node (4) at (4, -2.5) {};
    \node (5) at (-3.75, -2.5) {};
	\node (6) at (-4, 2.5) {};
	\node (7) at (4, 2.5) {};
	\node (8) at (0, 3) {};
	\node [label={$a$}] (9) at (0, 2.5) {};
	\node [label={left:$0$}] (21) at (-2.5, 0) {};
	\node [label={below:$-a$}] (22) at (0, -2.5) {};
	\node (23) at (1.5, 1.5) {};
	\node [label={right:$-1$}] (24) at (4, 1.5) {};
	\node (25) at (1.5, 0.5) {};
	\node [label={right:$-1$}] (26) at (4, 0.5) {};
	\node (27) at (1.5, -1.5) {};
	\node [label={right:$-1$}] (28) at (4, -1.5) {};
	\node (29) at (3, -0.35) {$\vdots$};
	\node [label={$X_a$}] (30) at (0, -5) {};
	% edges
	\draw (0.center) to (1.center);
	\draw (2.center) to (3.center);
	\draw (4.center) to (5.center);
	\draw (6.center) to (7.center);
	\draw (23.center) to (24.center);
	\draw (25.center) to (26.center);
	\draw (27.center) to (28.center);
	% brace
	\draw [decorate,decoration={brace,amplitude=5pt},xshift=-5pt,yshift=0pt]
(1.5,-1.5) -- (1.5,1.5) node [black,midway,xshift=-1em] {$b$};\
\end{tikzpicture}
\hspace{1cm}
\begin{tikzpicture}

    \node (0) at (-2.5, 4) {};
	\node (1) at (-2.5, -4) {};
	\node (2) at (2.5, 4) {};
	\node [label={below:$-a$}] (3) at (2.5, -4) {};
	\node (4) at (4, -2.5) {};
    \node (5) at (-3.75, -2.5) {};
	\node (6) at (-4, 2.5) {};
	\node (7) at (4, 2.5) {};
	\node (8) at (0, 3) {};
	\node [label={$b$}] (9) at (0, 2.5) {};
	\node [label={left:$0$}] (21) at (-2.5, 0) {};
	\node [label={below:$-b$}] (22) at (0, -2.5) {};
	\node (23) at (1.5, 1.5) {};
	\node [label={right:$-1$}] (24) at (4, 1.5) {};
	\node (25) at (1.5, 0.5) {};
	\node [label={right:$-1$}] (26) at (4, 0.5) {};
	\node (27) at (1.5, -1.5) {};
	\node [label={right:$-1$}] (28) at (4, -1.5) {};
	\node (29) at (3, -0.35) {$\vdots$};
	\node [label={$X_b$}] (30) at (0, -5) {};
	% edges
	\draw (0.center) to (1.center);
	\draw (2.center) to (3.center);
	\draw (4.center) to (5.center);
	\draw (6.center) to (7.center);
	\draw (23.center) to (24.center);
	\draw (25.center) to (26.center);
	\draw (27.center) to (28.center);
	% brace
	\draw [decorate,decoration={brace,amplitude=5pt},xshift=-5pt,yshift=0pt]
(1.5,-1.5) -- (1.5,1.5) node [black,midway,xshift=-1em] {$b$};\
\end{tikzpicture}
}
\]

 These admits a toric degeneration to $\mb{P}(1, b, ab-1)$ and $\mb{P}(1, a, ab-1)$ respectively. We only consider the case of $X_a$ as $X_b$ is completely symmetric. We see that the we can smooth it by taking the $b$th Veronese embedding and  getting $\mb{P}_{u,\,v,\,w,\,t}(1, 1, ab-1, a)$ with the relation $uw = t^b$. This admits a smoothing giving us the surface lies as $X_{ab} \subset \mb{P}(1,\, 1, \, ab-1, \, a)$. We get the following formula for the anticanonical degree of $X_a$:
 \[
 -K_{X_a}^2 =8 - b + a \left( 1- \frac{b+1}{ab-1} \right) ^2 + b \left(1- \frac{a+1}{ab-1} \right)^2 -2\left(1- \frac{a+1}{ab-1}\right)\left(1- \frac{b+1}{ab-1}\right)
\] 
We will show that this admits a cascade of length $a+2$. The first $a$ terms are easy to describe as we see that these admit a toric degeneration to $X_\Sigma$ with $\Sigma$ being the fan with rays $(-1, b), \, (-1, 0), \, (a, -1), \, (a-u, -1)$, where $u$ is the number of blowups. This has an $A_{b-1}$ singularity and an $A_{u-1}$ singularity.
Via Cox rings this can be viewed as $\mb{C}^4_\{x, y,z,t\}$ with a quotient 
\[
\begin{blockarray}{cccc}
	x & y & z & t \\[4pt]
      \begin{block}{(cccc)}
		u & 0 & bu - (ab-1) & ab-1 \\
		1 & ab-1 & b & 0 \\
      \end{block}
\end{blockarray}
\]
Taking the Veronese embedding of degree $u$ in the variables $x, \, z,\,t$ gets us the coordinates $z^u, \, t^u, \, zt$. And to smooth the $A_{b-1}$ singularity we take the $b$ Veronese embedding of the variables $x^b, \, y^b, \, xy$. Once again we can smooth this out. This gives us the surface as complete intersection with weights 
$\begin{matrix} b & b^2u \\ ab & u \end{matrix}$
inside the toric variety with weights

\[
\begin{blockarray}{ccc ccc}
	x^b & y^b & z^u & xy & t^u & t^{bu}z \\[4pt]
      \begin{block}{(ccc ccc)}
		b & 0 & b(bu - (ab-1) ) & 1 & ab-1 & b^2\\
		1 & ab-1 & u & a & 0 & 1\\
      \end{block}
\end{blockarray}
\]

 The $(a+1)$st blowup admits a toric degeneration to $(-1, b), \, (-1, -1), \, (a, -1)$. The toric degeneration of the $a+2$'nd blowup is a bit more finicky and goes on a case by case analysis. We note that this surface still has a boundary of a curve  $C \in -K_X$. This strict transform $\wt{C} \subset Y$ has self intersection $0$ as it started of as an $a+2$ curve and has been blown up $a+2$ times. Hence $X$ admits a degeneration as it is a \LJ. To see the cascade result we note that if you blow up the surface $X_a$ times at points $P_1 \dots P_a$. To each of these points there is a unique fiber going through it $F_i$. The strict transform of these fibers after blowing up is a $-1$-curve going through the $-a$ curve. Hence after blowing $X_a$ and $X_b$ respectively $a$ and $b$ times we get a surface which has as a boundary three curves with self intersection $0, -a, -b$ and in both cases you have your $a$ and your $b$ curves have $a$ and $b$ minus one curves intersecting them respectively. Hence they are isomorphic. 
 
 
 We note that we have made in the above calculations no effort to show that the elements in the cascade are \ldp\ surfaces. However it is not hard to show assume we are blowing up $a+2$ points giving surface $X_3$. This has minimal resolution $Y_3$. The class group of $Y_3$ is generated by the curves $D_1, \, D_2, \, D_3, \, D_4, \, E_1^0, \dots E_b^0, \, E_1^1, \dots E_{a+2}^1$. Here the $D_i$ for a cycle such that $\sum D_i \in |-K_{Y_3}|$. These have self intersections $-a, \, -b, \, -1, \, -1$ respectively. Here $D_3$ was a curve of degree $a$  on $\mb{F}_a$ blown up $a+1$ times and $D_4$ was a fiber on which a point has been blown up. The $E_i^0$ are $-1$-curves intersecting the $-b$ curve. The $E_i^1$ are floating $-1$-curves. We wish to show $-K_{X_3}$ is ample. We note that showing $-K_{X_3} \dot C > 0$ for all $C$ generating the class group would suffice. We note that the curves $D_1, \, D_2$ are contracted when sent to $X_3$. We note that $-K_{X_3} \dot E_i^0 = -K_{X_a} \dot E_i^0$ as we are blowing up points not on these curves. Then $-K_{X_3} \dot E_i^0 = 1$ as these are floating $-1$-curves. Finally to see that $-K_{X_3} \dot D_3 > 0$ we note that, when pushed forwards to $X_3$, it only goes through the one singularity on $X_3$ with multiplicity $-1$. This is because on $Y_3$ it is only intersecting the $-a$ curve transversely. Hence $-K_{X_3} \dot D_3 = 1 + d_a$ where $d_a$ is the discrepancy of the $-a$ curve. Via log terminality we have $d_a > -1$. Hence the product is greater than 0. The argument for the curve is exactly the same with $d_a$ replaced with $d_b$. From this we see $X_3$ is a \ldp, hence every surface in the cascade is a \ldp\ surface.
\end{proof}

This structure of the cascade can be put in more general terms. 

\begin{thm}
Given a singularity with small discrepancy such that the minimal resolution is $a_1, \dots a_n$, this has at most $n$ basic surfaces from the previous theorem, possibly less via symmetry. Let $X_i$ be the surface constructed from $X_{a_i}$, if $i \neq 1,\, n$ then this admits a cascade of length $a_i + 3 - n$, if $i = 1, \, n$ then the cascade is of length $a_i + 4 - n$. 

In addition we can describe the shape of the cascade. In the case when the singularities are of the form $\frac{1}{p}(1,1)$ the cascades have been classified by \cite{CP} and the above example. We explained the case of the singularity of length 2 above. In the case of your singularity having length 3, then the cascade looks like:
\[
% https://tikzcd.yichuanshen.de/#N4Igdg9gJgpgziAXAbVABwnAlgFyxMJZABgBpiBdUkANwEMAbAVxiRAA0B9YARgF8AesRB9S6TLnyEUPclVqMWbLr0E8RYkBmx4CRAExzq9Zq0QgAOhagQcCUeJ1SiAZiMLTy7vwHA6nHgBafg1HST0UNx55EyVzLn1ff30+UK0JXWkSUn0YxTMObhShNO1wrNlc43yvYGL1B3SnCORDKo84y2tbe00yzNccvM94osE-Tn1g1Mb+5xQAViHqkcKGvoz55AA2ZY6ChNLNlrIAFmHOlRdBYVnjitJzlcvua4F1sIGUQyf9tisbHYjs0sm5frEDmMkpwXNNgeUiAB2PYQrwuETyGBQADm8CIoAAZgAnCAAWyQshAOAgSDIf3MVjQAAssAF3iBqAw6AAjGAMAAK9zYRKw2KZODSxLJSEMVJpiEpqIZFmZrJ4An0HJAXN5AqF5hFYoljSl5MQbjlMueBUZLLZEyCKS1Or5gpBwtF4slJLNp2o1KQAA5rf8VXbEhMpiFOTzXfrtTACcbNKakLtLYglvSuqrJuyY7q3QiDZ7k4SfUhkRn00qc+GNc7Y3r3SWjd7pYhgxmq7XbayI8k4QW4y2QIavSaK53-fKAJwh5UAFSZMBwdEbhfjDETZZAqcQ84zFt7YbV0KC0e1TaLXzHpfbZsPAYVfuzfZh0Nhl5dzeLCaTD4UnSz6Hie-J2uow6-re467vuPCys+PCKjUyq5sAbyQVem6jrBgEKhaSGyie6FvJqUE3vMd5tpOHY8K+SHHqhdashh4z+LCToUfGeG0WaPBZkhdJgeGG4jn+vEUHwQA
\begin{tikzcd}
X_{1}^0 & X_{1}^1 \arrow[l, "\phi_1^1"']   & \dots \arrow[l, "\phi_1^2"']   & X_{1}^{a_1-1} \arrow[l, "\phi_1^{a_1-2}"']                      &  &                                                                    &                          &                          \\
        &                                  &                                & X_2^{a_2} \arrow[d, "\phi_2^{a_2-1}"]                           &  &                                                                    &                          &                          \\
X_{2}^0 & X_{2}^1 \arrow[l, "\phi_2^1"']   & \dots \arrow[l, "\phi_2^2"']   & X_{2}^{a_2-1} \arrow[l, "\phi_2^{a_2-1}"'] \arrow[rr, "\Theta"] &  & X_1 \arrow[lluu, "\phi_1^{a_1-1}"'] \arrow[lldd, "\phi_3^{a_3-1}"] & X_2 \arrow[l, "\Phi_1"'] & X_3 \arrow[l, "\Phi_2"'] \\
        &                                  &                                &                                                                 &  &                                                                    &                          &                          \\
X_{3}^0 & X_{3}^1 \arrow[l, "\phi_{3}^1"'] & \dots \arrow[l, "\phi_{3}^2"'] & X_{2}^{a_3-1} \arrow[l, "\phi_{3}^{a_3-2}"']                    &  &                                                                    &                          &                         
\end{tikzcd}
\]
If the singularity is length 4 we get two cascades with length $a_i + 3 - n$ for $ i = 2, \, 3$. These do not intersect at all, and no two surfaces in these cascades are isomorphic. In the case of $i = 1, \, 4$ you have 
\[
% https://tikzcd.yichuanshen.de/#N4Igdg9gJgpgziAXAbVABwnAlgFyxMJZABgBpiBdUkANwEMAbAVxiRAA0B9YARgF8AesRB9S6TLnyEUPclVqMWbLr0E8RYkBmx4CRAExzq9Zq0QgAOhYDGUCDgSjxOqUQDMRhaeWceA4HScWAC0-BrOknooACykPPImSubs4VoSutIkpPoJimYcnNEC6k5pLlHIsjnGeT5F+qnakZmG1V5JljZ2Do3prigebYn5XEUBhaF8IvIwUADm8ESgAGYAThAAtkhkIDgQSLLt+VZoABZYvkKpa5sH1HtIhkdsJ+eXJZo3W4hPD4gez3MrwufnGPGC+impS+SABf1igM6ZxB-kCITC0PW3wArPd9ogAGw1bxAizIwpXTG3Ql4pAAdmJHWBFI+KyxSAAHLTEAiGHQAEYwBgABT6URADBgyxwIEZxzJbzGgWik2u7MQDN2+K5iOZSomkOmfCAA
\begin{tikzcd}
X_{1}^0 \arrow[r, "\phi_1^0"] & X_{1}^1 \arrow[r, "\phi_1^1"] & \cdots \arrow[r, "\phi_1^{a_1-2}"] & X_1^{a_i-1} \arrow[rd, "\phi_1^{a_i-1}"] &   \\
                              &                               &                                    &                                          & X \\
X_4^1 \arrow[r, "\phi_4^0"]   & X_4^2 \arrow[r, "\phi_4^1"]   & \cdots \arrow[r, "\phi_4^{a_4-2}"] & X_4^{a_4-1} \arrow[ru, "\phi_4^{a_4-1}"] &  
\end{tikzcd}
\]
If the length of the singularity is greater than 4 every surface constructed by Theorem~\ref{ThmOnSing} will have their cascades never intersecting, unless the whole cascades are isomorphic to each other.
\end{thm}
\begin{proof}
Assume the length of the singularity is not of length one. We start by noting that $Y_i$, the minimal resolution of $X_i$, has a cycle $C_1 \dots C_n \in -K_{Y_i}$ with intersections $-a_1, \dots -a_n, a_i+4-n$. If $a_i+4 -n \leq -2$ then this is clearly not the minimal resolution of a \ldp\ with cyclic quotients. If $a_i+4-n = -1$ then we have a $-1$-curve intersecting the singularity twice. 



So now assume $a_i + 4-n > -1$, then if we blowup the surface $a_i + 4 -n $ times we can assume that all the points lie on this curve. Clearly if we blow up $a_i + 5 - n$ times we have $-1-$ curve intersecting the singularity twice. If we blowup point $P_1, \dots, P_k$, with $k \leq a_i + 4 - n$, to obtain a surface $X'$. We wish to show $-K_{X'}$ is ample, note $-K_{X'}\cdot C = -K_{X} - \sum{E_i} \cdot C$. We note that the Weil group of $X$ is generated by the toric boundary and the curves which are occurring as blowups on the boundary. In the latter case clearly our product is unchanged and hence it is still greater than zero. As our singularity has length greater than 2, there are at most only two components to the toric boundary. The first is the strict transform of the $l$ curve. 
\end{proof}
\begin{cor}
Let $S$ be a singularity with small discrepancy, $-a_1, \dots , -a_n$ be the self intersection of the resolutions. Then if $n \geq \max (a_i) + 5 $. Then there exists no \ldp\ with only singularities of type $S$.
\end{cor}
\begin{rem}
It is fully possible for both $X_1$ and $X_2$ to exist but one of the $X_{a_i}^0$ to not exist. For example consider a singularity with resolution $-3, \, -8, \, -2, \, -2, \, -2, \, -2, \, -2, \, -2, \, -3$. There will be a surface $X$ such that the resolution will have a map to $\mb{F}_8$ but there will be no surface with a map from its resolution to $\mb{F}_3$.
\end{rem}

\end{document}