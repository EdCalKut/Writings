\documentclass[12pt]{amsart}

\usepackage{geometry}  
\geometry{letterpaper} 
\usepackage{graphicx}
%\usepackage[backend=bibtex]{biblatex}
\usepackage{array}
\usepackage{amssymb}
\usepackage{amsmath}
\usepackage{amsthm}
\usepackage{graphicx}
\usepackage[parfill]{parskip} 
\usepackage[utf8]{inputenc}
\usepackage[english]{babel}
\usepackage{tikz}
\usepackage{tikz-cd}
\usepackage[noend]{algpseudocode}
\usepackage{caption}
\usepackage{subcaption}
\usepackage{fancyhdr}
\usepackage{enumitem}
\usepackage[super]{nth}
\usepackage{pstricks}



\usepackage[colorlinks=true,linkcolor=blue]{hyperref}


\pagestyle{fancy}
\lhead{}
\chead{}
\rhead{}
\lfoot{}
\cfoot{\thepage}
\rfoot{}
\renewcommand{\headrulewidth}{0pt}
\setlength{\footskip}{50pt}

\makeatletter
\def\BState{\State\hskip-\ALG@thistlm}
\makeatother

\theoremstyle{definition}
\newtheorem{thm}{Theorem}[section]
\theoremstyle{definition}
\newtheorem{cor}[thm]{Corollary}
\theoremstyle{definition}
\newtheorem{prop}[thm]{Proposition}
\theoremstyle{definition}
\newtheorem{dfn}[thm]{Definition}
\theoremstyle{definition}
\newtheorem{lem}[thm]{Lemma}
\theoremstyle{definition}
\newtheorem{ex}[thm]{Example}
\theoremstyle{definition}
\newtheorem{conj}[thm]{Conjecture}
\theoremstyle{definition}
\newtheorem*{rem}{Remark}

\newcommand{\Rom}[1]
    {\MakeUppercase{\romannumeral #1}}
\newcommand{\C}[1]{(\mathbb{C}^*)^#1}
\newcommand{\ldp}{log del Pezzo}
\newcommand{\mb}[1]{\mathbb{#1}}
\newcommand{\Hi}{Hirzebruch surface }
\newcommand{\minres}{minimal resolution }
\newcommand{\LJ}{Looijenga pair }
\newcommand{\ra}{\rightarrow}
\newcommand{\spl}{\text{SL}_2 (\mathbb{C})}
\newcommand{\gl}{\text{GL}_2 (\mathbb{C})}
\newcommand{\pgl}{\text{PGL}_2 (\mathbb{C})}

\graphicspath{ {images/} }

\begin{document} 

This thesis solves a range of classification problems for singular surfaces.


Throughout this thesis we consider varieties $X$ with $-K_X$ ample and various restrictions on the singularities. These are particular instances of Fano varieties. A two dimensional Fano variety is called a \ldp\ surface, see ~\ref{beginnersdef}. Smooth log del Pezzo surfaces were classified in the early twentieth century. These surface are all of the form $\mb{P}^2$ blown up in $k$ points where $k <9 $, or the exceptional case $\mb{P}^1 \times \mb{P}^1$. Such an elegant classification does not exist in the case of \ldp\ surfaces with singularities; as represented by \cite{Reid, etc....}. A lot of work has been done on extending this classification to singular surfaces. In particular recent approaches have been interested in using of machinery of toric degeneration's. This technique involves constructing a family $\mathcal{X}$ over $\mb{A}^1$ such that the fiber over $0$ is a normal surface that contains $\C{2}$ as a dense subvariety for which the natural action of the torus extends to the variety.  



 Work of \cite{CH} and \cite{AC} have established a one to one correspondence between \ldp\ surfaces with $h^0(-K_X) \neq 0$ and toric degenerations subject to assumptions on the singularities of the surface. It has been conjectured that this one to one correspondence extends to other classes of singularities beyond those considered in the above papers. There has also been work by a variety of authors, \emph{Cavey et.al} \cite {CSasd}, trying to bound the types of singularities that can occur on such toric degenerations. 
 
 
 
%We discuss these results in the following way:


In the case of surfaces it is interesting to study the log del Pezzos with log terminal singularities. In the full generality, a log terminal surface singularity is a group quotient by a subgroup of $\gl$. The case when the subgroup is cyclic is particularly important, and we refer to these as cyclic quotients. In the case of cyclic quotient singularities it has been conjectured that these admit toric degeneration.


This thesis is about log del Pezzo surfaces. The formal definition is:
\begin{dfn}\label{beginnersdef}
A log del Pezzo surface is a normal two dimensional variety over $\mb{C}$ which has only log terminal singularities and has $-K_X$ ample.
\end{dfn}
see Definition~\ref{Log terminal Singularity} for a definition of log terminal singularities.

Our motivating aim is the classification of such surfaces.
This is an absolutely hopeless task in full generality.
Nevertheless, we can classify special cases as follows.

\subsection{Log del Pezzo surfaces of complexity 1}
The Gorenstein index of a singularity $S$ is the smallest value $n$ such that $n K_S$ is Cartier (see Def~\ref{}). We define the Gorenstein index of a surface $X$ to be the smallest value $n$ such that $nK_X$ is Cartier.
For any given $i \in \mathbb{N}$, the set of deformation
families of log del Pezzo surfaces $X$ with Gorenstein index $i_X=i$
is finite \cite{f}.

It is worth noting that the number of families increases enormously as the Gorenstein
index of the surface increases. For example, only in the toric case, 
the start of the classification is
\[
\begin{array}{|c|c|}
\hline\\
\text{index $i$} & \text{number of toric surfaces}\\ \hline
1 & 16 \\
2 & 30 \\
3 & 99  \\
4 & 91  \\
5 & 250 \\
6 & 379 \\
7 & 429 \\
8 & 307 \\
9 & 690 \\
10 & 916 \\
\hline
\end{array}
\]

We consider these surfaces from three different and related points of view.

In particular, we classify log del Pezzo surfaces that admit a $\mathbb{C}^\times$ action.
In this thesis we give an algorithm to classify log del Pezzo surfaces that admit a $\mathbb{C}^\times$ action and which have only log terminal singularities. 


A variety $X$ of dimension $n$ equipped with an action of a torus of dimension $n-k$ is referred to as a variety of complexity $k$; see Definition~\ref{forwardref} for the precise definition. To illustrate the notion, note first that a toric variety $X$ has an action of its $n$-dimensional `big torus' $T\subset X$, and equipped with this action $X$ is a variety of complexity~0.
One could also give consider $X$ equipped with the natural action of a $k$-dimensional
subtorus $T'\subset T$, and then $X$ is a variety of complexity~$k$. (See \S\ref{asdf}.)

However, there are many varieties of complexity $k<n$ whose torus action does not extend to a toric variety. This is one of the main themes of this thesis: we study
and classify surfaces of complexity~1 that are not toric.

In this way, complexity provides a way of grading the difficulty of a classification problem. Significant progress has been made on this problem before: S\"{u}ss \cite{Suss} classifies log del Pezzo surfaces admitting a $\mathbb{C}^\times$ action which have picard rank one and Gorenstein index less than~3. Huggenberger \cite{Huggenberger} classifies log del Pezzo surfaces of complexity~1 that have index 1 and arbitrary picard rank. Ilten, Mishna and Trainor \cite{IMT} recover the same classification and extend it into higher dimension. The methods and language used are broadly the same (though, in the language of toric geometry, it varies whether papers work in the lattice $N$ or its dual lattice $M$), though Huggenberger exploits Hausen's anticanonical complex technology to describe the Cox ring in detail. 

We extend these results by presenting an algorithm that classifies
log del Pezzo surfaces of complexity 1 with given index.
The algorithm works and \textbf{terminates for any} index, 
though since the index is an unbounded invariant, there is no hope of 
a closed-form classification via methods of this type for all such del Pezzo with a torus action.



\subsection{Bounded singularity content of log del Pezzo surfaces}

We can consider log del Pezzos from a completely different point of view. Rather than considering the global invariants, we can consider the local invariants of the singularities. 
It follows from the definition (Def \ref{}) that the singularities
are all finite quotient singularities, but this class of singularities itself is an infinite set. 
The {\em discrepancies} associated to a singularity (see Def~\ref{}) form a measure of
its complexity expressed as a collection of rational numbers, one for each curve in a resolution. 
When these numbers are small, but still greater than $-1$, the singularity may be regarded as `more complicated'. 
However surfaces that have only these more complicated singularities can be classified explicitly. Informally,
the basic reason is that it is hard to impose many of these singularities onto a single surface.

These conditions naturally arise as soon as you start to consider singularities in families. The first place this was considered was in \cite{CP} where they considered the case of $\frac{1}{p}(1,1)$ singularities, where $p \geq 5$. We extend this by


\begin{thm}[= Theorem~\ref{ref to main appearance in text}]
Let $X$ be a surface with singularities of only small discrepancy then $X$ has at most one singularity except for one sporadic family. All of these log del Pezzo surfaces admit a toric degeneration.
\end{thm}

This reproves the results of \cite{CP} who classified log del Pezzo surfaces with only $\frac{1}{p}$ singularities and extends results in \cite{Cuzzo} who classified surfaces with $\frac{1}{5}(1,2)$  and $\frac{1}{p}(1,1)$ singularities.

We also consider how the cascade of these surfaces behaves. This notion was introduced in \cite{RS} and is essentially asking for the birational relations between the surfaces. We proof that once our singularity is sufficiently complicated then you get the folllowing series of birational relations
\[
% https://tikzcd.yichuanshen.de/#N4Igdg9gJgpgziAXAbVABwnAlgFyxMJZABgBpiBdUkANwEMAbAVxiRAA0B9YARgF8AesRB9S6TLnyEUPclVqMWbLr0E8RYkBmx4CRAExzq9Zq0QgAOhagQcCUeJ1SiAZiMLTy7vwHA6nHgBafg1HST0UNx55EyVzLn1ff30+UK0JXWkSUn0YxTMObhShNO1wrNlc43yvYGL1B3SnCORDKo84y2tbe00yzNccvM94osE-Tn1g1Mb+5xQAViHqkcKGvoz55AA2ZY6ChNLNlrIAFmHOlRdBYVnjitJzlcvua4F1sIGUQyf9tisbHYjs0sm5frEDmMkpwXNNgeUiAB2PYQrwuETyGBQADm8CIoAAZgAnCAAWyQshAOAgSDIf3MVjQAAssAF3iBqAw6AAjGAMAAK9zYRKw2KZODSxLJSEMVJpiEpqIZFmZrJ4An0HJAXN5AqF5hFYoljSl5MQbjlMueBUZLLZEyCKS1Or5gpBwtF4slJLNp2o1KQAA5rf8VXbEhMpiFOTzXfrtTACcbNKakLtLYglvSuqrJuyY7q3QiDZ7k4SfUhkRn00qc+GNc7Y3r3SWjd7pYhgxmq7XbayI8k4QW4y2QIavSaK53-fKAJwh5UAFSZMBwdEbhfjDETZZAqcQ84zFt7YbV0KC0e1TaLXzHpfbZsPAYVfuzfZh0Nhl5dzeLCaTD4UnSz6Hie-J2uow6-re467vuPCys+PCKjUyq5sAbyQVem6jrBgEKhaSGyie6FvJqUE3vMd5tpOHY8K+SHHqhdashh4z+LCToUfGeG0WaPBZkhdJgeGG4jn+vEUHwQA
\begin{tikzcd}
X_{1}^0 & X_{1}^1 \arrow[l, "\phi_1^1"']   & \cdots \arrow[l, "\phi_1^2"']   & X_{1}^{a_1-1} \arrow[l, "\phi_1^{a_1-2}"']                      &  &                                                                    &                          &                          \\
        &                                  &                                & &  &                                                                    &                          &                          \\
X_{2}^0 & X_{2}^1 \arrow[l, "\phi_2^1"']   & \cdots \arrow[l, "\phi_2^2"']   & X_{2}^{a_2-1} \arrow[l, "\phi_2^{a_2-1}"']  &  &\arrow[ll, "\phi_2^{a_2}"]  X_1 \arrow[lluu, "\phi_1^{a_1-1}"'] \arrow[lldd, "\phi_3^{a_3-1}"] & X_2 \arrow[l, "\Phi_1"'] & X_3 \arrow[l, "\Phi_2"'] \\
        &                                  &                                &                                                                 &  &                                                                    &                          &                          \\
X_{3}^0 & X_{3}^1 \arrow[l, "\phi_{3}^1"'] & \cdots \arrow[l, "\phi_{3}^2"'] & X_{2}^{a_3-1} \arrow[l, "\phi_{3}^{a_3-2}"']                    &  &                                                                    &                          &                         
\end{tikzcd}
\]

In certain cases not all branches of this diagram may exist, and examples of this are provided in \ref{Length two sing}. In addition we provide simple examples outside of small discrepancy where toric degeneration do not exist.

\subsection{Smoothings of log extremal extractions}

To fit with the ongoing interest in toric degenerations, we study the case of a log terminal cyclic extractions from a given singularity. These are maps $f: \: Y \rightarrow X$ with relative Picard rank one, such that both $X$ and $Y$ only have cyclic quotient singularities along with other technical conditions. We proof the following:

\begin{thm}
Let  $f: \: Y \rightarrow X$ be a cyclic extraction in dimension two then both $Y$ admits a toric degeneration which $Y_\Sigma$ which extends map $f$ to $f_\Sigma : \: Y \rightarrow X$
\end{thm}
We then charecterise these possible toric degenerations, and extend this in part to higher dimension. We also provide several examples of how this can be applied to the global case. In addition in dimension greater than or equal to three we discuss how this gives explicit equations for every single possible deformation of the toric variety. In addition we show how this relates with notion of focus-focus singularities and the SYZ fibration in dimension 2.


\section{Map of the thesis}

1 - Intro

2 - Background
	- Log del Pezzos
	- Polyhedral divisors
	- Looijenga pairs

3 - Bounded Sing Content (Corti Heuberger, Reproves(Cavey Prince), Cuzzocolli)
	- Small Sing
	- Outside Bounded

4 - LDP complexity One - Redervies and generalises (Huggenberger,  Suss, Ilten..)
	- Algorithm 1 
	- Algorithm 2 (Smarter)

5 - Smoothings - Related to Laurent invesrion but strictly weaker
	- Two dimensional
	- $n$ dimensional
	- global example
	- charecterisation of when smoothing exists (purely combinatorial)

6 - Complexity One Fanos Terminal - Rederives (Kasprzyk 03, Huggenberger-Nicholussi et.al) and then expands.


\end{document}